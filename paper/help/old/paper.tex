% Template for PLoS
% Version 3.5 March 2018
%
% % % % % % % % % % % % % % % % % % % % % %
%
% -- IMPORTANT NOTE
%
% This template contains comments intended
% to minimize problems and delays during our production
% process. Please follow the template instructions
% whenever possible.
%
% % % % % % % % % % % % % % % % % % % % % % %
%
% Once your paper is accepted for publication,
% PLEASE REMOVE ALL TRACKED CHANGES in this file
% and leave only the final text of your manuscript.
% PLOS recommends the use of latexdiff to track changes during review, as this will help to maintain a clean tex file.
% Visit https://www.ctan.org/pkg/latexdiff?lang=en for info or contact us at latex@plos.org.
%
%
% There are no restrictions on package use within the LaTeX files except that
% no packages listed in the template may be deleted.
%
% Please do not include colors or graphics in the text.
%
% The manuscript LaTeX source should be contained within a single file (do not use \input, \externaldocument, or similar commands).
%
% % % % % % % % % % % % % % % % % % % % % % %
%
% -- FIGURES AND TABLES
%
% Please include tables/figure captions directly after the paragraph where they are first cited in the text.
%
% DO NOT INCLUDE GRAPHICS IN YOUR MANUSCRIPT
% - Figures should be uploaded separately from your manuscript file.
% - Figures generated using LaTeX should be extracted and removed from the PDF before submission.
% - Figures containing multiple panels/subfigures must be combined into one image file before submission.
% For figure citations, please use "Fig" instead of "Figure".
% See http://journals.plos.org/plosone/s/figures for PLOS figure guidelines.
%
% Tables should be cell-based and may not contain:
% - spacing/line breaks within cells to alter layout or alignment
% - do not nest tabular environments (no tabular environments within tabular environments)
% - no graphics or colored text (cell background color/shading OK)
% See http://journals.plos.org/plosone/s/tables for table guidelines.
%
% For tables that exceed the width of the text column, use the adjustwidth environment as illustrated in the example table in text below.
%
% % % % % % % % % % % % % % % % % % % % % % % %
%
% -- EQUATIONS, MATH SYMBOLS, SUBSCRIPTS, AND SUPERSCRIPTS
%
% IMPORTANT
% Below are a few tips to help format your equations and other special characters according to our specifications. For more tips to help reduce the possibility of formatting errors during conversion, please see our LaTeX guidelines at http://journals.plos.org/plosone/s/latex
%
% For inline equations, please be sure to include all portions of an equation in the math environment.
%
% Do not include text that is not math in the math environment.
%
% Please add line breaks to long display equations when possible in order to fit size of the column.
%
% For inline equations, please do not include punctuation (commas, etc) within the math environment unless this is part of the equation.
%
% When adding superscript or subscripts outside of brackets/braces, please group using {}.
%
% Do not use \cal for caligraphic font.  Instead, use \mathcal{}
%
% % % % % % % % % % % % % % % % % % % % % % % %
%
% Please contact latex@plos.org with any questions.
%
% % % % % % % % % % % % % % % % % % % % % % % %

\documentclass[10pt,letterpaper]{article}
\usepackage[top=0.85in,left=2.75in,footskip=0.75in]{geometry}

% amsmath and amssymb packages, useful for mathematical formulas and symbols
\usepackage{amsmath,amssymb}

% Use adjustwidth environment to exceed column width (see example table in text)
\usepackage{changepage}


% textcomp package and marvosym package for additional characters
\usepackage{textcomp,marvosym}

% cite package, to clean up citations in the main text. Do not remove.
% \usepackage{cite}

% Use nameref to cite supporting information files (see Supporting Information section for more info)
\usepackage{nameref,hyperref}

% line numbers
\usepackage[right]{lineno}

% ligatures disabled
\usepackage{microtype}
\DisableLigatures[f]{encoding = *, family = * }

% color can be used to apply background shading to table cells only
\usepackage[table]{xcolor}

% array package and thick rules for tables
\usepackage{array}

% create "+" rule type for thick vertical lines
\newcolumntype{+}{!{\vrule width 2pt}}

% create \thickcline for thick horizontal lines of variable length
\newlength\savedwidth
\newcommand\thickcline[1]{%
  \noalign{\global\savedwidth\arrayrulewidth\global\arrayrulewidth 2pt}%
  \cline{#1}%
  \noalign{\vskip\arrayrulewidth}%
  \noalign{\global\arrayrulewidth\savedwidth}%
}

% \thickhline command for thick horizontal lines that span the table
\newcommand\thickhline{\noalign{\global\savedwidth\arrayrulewidth\global\arrayrulewidth 2pt}%
\hline
\noalign{\global\arrayrulewidth\savedwidth}}


% Remove comment for double spacing
%\usepackage{setspace}
%\doublespacing

% Text layout
\raggedright
\setlength{\parindent}{0.5cm}
\textwidth 5.25in
\textheight 8.75in

% Bold the 'Figure #' in the caption and separate it from the title/caption with a period
% Captions will be left justified
\usepackage[aboveskip=1pt,labelfont=bf,labelsep=period,justification=raggedright,singlelinecheck=off]{caption}
\renewcommand{\figurename}{Fig}

% Use the PLoS provided BiBTeX style
% \bibliographystyle{plos2015}

% Remove brackets from numbering in List of References
\makeatletter
\renewcommand{\@biblabel}[1]{\quad#1.}
\makeatother



% Header and Footer with logo
\usepackage{lastpage,fancyhdr,graphicx}
\usepackage{epstopdf}
%\pagestyle{myheadings}
\pagestyle{fancy}
\fancyhf{}
%\setlength{\headheight}{27.023pt}
%\lhead{\includegraphics[width=2.0in]{PLOS-submission.eps}}
\rfoot{\thepage/\pageref{LastPage}}
\renewcommand{\headrulewidth}{0pt}
\renewcommand{\footrule}{\hrule height 2pt \vspace{2mm}}
\fancyheadoffset[L]{2.25in}
\fancyfootoffset[L]{2.25in}
\lfoot{\today}

%% Include all macros below

\newcommand{\lorem}{{\bf LOREM}}
\newcommand{\ipsum}{{\bf IPSUM}}


% tightlist command for lists without linebreak
\providecommand{\tightlist}{%
  \setlength{\itemsep}{0pt}\setlength{\parskip}{0pt}}


% Pandoc citation processing
\newlength{\cslhangindent}
\setlength{\cslhangindent}{1.5em}
\newlength{\csllabelwidth}
\setlength{\csllabelwidth}{3em}
\newlength{\cslentryspacingunit} % times entry-spacing
\setlength{\cslentryspacingunit}{\parskip}
% for Pandoc 2.8 to 2.10.1
\newenvironment{cslreferences}%
  {}%
  {\par}
% For Pandoc 2.11+
\newenvironment{CSLReferences}[2] % #1 hanging-ident, #2 entry spacing
 {% don't indent paragraphs
  \setlength{\parindent}{0pt}
  % turn on hanging indent if param 1 is 1
  \ifodd #1
  \let\oldpar\par
  \def\par{\hangindent=\cslhangindent\oldpar}
  \fi
  % set entry spacing
  \setlength{\parskip}{#2\cslentryspacingunit}
 }%
 {}
\usepackage{calc}
\newcommand{\CSLBlock}[1]{#1\hfill\break}
\newcommand{\CSLLeftMargin}[1]{\parbox[t]{\csllabelwidth}{#1}}
\newcommand{\CSLRightInline}[1]{\parbox[t]{\linewidth - \csllabelwidth}{#1}\break}
\newcommand{\CSLIndent}[1]{\hspace{\cslhangindent}#1}

\usepackage{rotating, graphicx, dcolumn, booktabs, caption, xcolor}


\usepackage{forarray}
\usepackage{xstring}
\newcommand{\getIndex}[2]{
  \ForEach{,}{\IfEq{#1}{\thislevelitem}{\number\thislevelcount\ExitForEach}{}}{#2}
}

\setcounter{secnumdepth}{0}

\newcommand{\getAff}[1]{
  \getIndex{#1}{University of Bristol and The Alan Turing Institute}
}

\begin{document}
\vspace*{0.2in}


% Title must be 250 characters or less.
\begin{flushleft}
{\Large
\textbf\newline{Title of submission to PLOS
journal} % Please use "sentence case" for title and headings (capitalize only the first word in a title (or heading), the first word in a subtitle (or subheading), and any proper nouns).
}
\newline
% Insert author names, affiliations and corresponding author email (do not include titles, positions, or degrees).
\\
Emmanouil Tranos\textsuperscript{\getAff{University of Bristol and The
Alan Turing Institute}}\textsuperscript{*}\\
\bigskip
\textbf{\getAff{University of Bristol and The Alan Turing
Institute}}University of Bristol, School of Geographical Sciences,
Bristol, UK; The Alan Turing Institute, London, UK\\
\bigskip
* Corresponding author: e.tranos@bristol.ac.uk\\
\end{flushleft}
% Please keep the abstract below 300 words
\section*{Abstract}
blah blah

% Please keep the Author Summary between 150 and 200 words
% Use first person. PLOS ONE authors please skip this step.
% Author Summary not valid for PLOS ONE submissions.
\section*{Author summary}
blah blah

\linenumbers

% Use "Eq" instead of "Equation" for equation citations.
\hypertarget{sec:1}{%
\section{Introduction}\label{sec:1}}

Geographers were always interested in how new technologies and
innovations diffuse across space and, importantly, how such
spatio-temporal processes can be modelled. The seminal contribution of
{[}1{]} is illustrative of this early interest. However, the torch of
exploring and modelling such processes had been passed to other
disciplines such as economics, business studies and sociology well
before the `cultural turn' of economic geography {[}2{]}. A potential
explanation of lack of studies exploring the diffusion of new and, more
specifically, digital technologies across \emph{both} space and time can
be attributed to the scarcity of relevant and granular enough data. As
{[}3{]} highlight, digital activities are hardly ever captured in
official data.

This paper aims to reinstate geographers' interest in how new and
digital technologies diffuse over space and time. It adopts a geographic
standpoint to explore and map the diffusion of web technologies in the
UK over space and time. It does so by employing a novel source of big
data which captures the active engagement with web technologies during
the \(1996-2013\) period.

The motivation for this paper lies in the fact that there are various
stakeholders who are interested in knowing how new digital technologies
diffused over space and time use this information to make predictions
regarding the diffusion of related future technologies. As per {[}4{]} a
variety of actors have a direct interest in gaining such knowledge
including network equipment suppliers; network operators, regulatory and
local authorities. And despite the spatial heterogeneity of such
diffusion mechanisms and the policy relevance, there are very limited
attempts in the literature to analyse the diffusion of new digital
technologies at a granular level.

the paper answers an empirical question: how did a

{[}3{]} Because new digital activities are rarely---if ever---captured
in official state data, researchers must rely on information gathered
from alternative sources. With this in mind, the overarching aim of this
article is outlining an approach for analyzing the ``new spaces and
geographies'' of digital phenomena and practices.

{[}4{]} There are numerous agents who would benefit from accurate
predictions of introduction times. These include: suppliers of network
equipment for production planning; suppliers of mobile phones who need
to have inventories of appropriately configured handsets; international
network planners who need to provide sufficient bandwidth for
communications between different countries; network operators and
regulatory authorities.

Classifying adopters of a technology along a time axis was pioneered by
Rogers (1962), who categorized adopters into five groups as innovators,
early adopters, early majority, late majority and laggards.

An a priori approach groups similar countries into segments, where
similarity is defined using economic, social, or political covariates.
Examples include Hofstede (2001) and Lee (1990). An a posteriori
analysis considers segments based on realized market behaviour, for
example see Sood et al.~(2009). Some studies investigate changing
segment membership over time, see Cannon and Yaprak (2011).

{[}5{]} Hägerstrand (1952, 1967) was the first scholar to intensively
analyze spatial diffusion pro cesses. Based on the empirical
observations of the regional spread of process innovations, he
hypothesized that new technology is first implemented in the `center' --
the large agglom erations -- and then moves `down' the spatial
hierarchy, the last stop being the `periphery', i.e.~remote and sparsely
populated regions. There are at least two factors that may explain this
pattern: first, the transfer of tacit knowledge via face-to-face contact
and second, the absorptive capacity of a region.

{[}6{]} The concept of absorptive capacity has also been applied in the
TIS context in terms of the con-struction of capacity needed to transfer
and implement a new technology in the receiving country (cf.Van Alphen,
2011), with repercussions for the nature of system functions

{[}7{]} larger local technological niches (at least from a demand
perspective) are more likely to be formed in urban areas than in rural
ones, which is in line with the work of Ha¨gerstrand (1965a,b). However,
size is not the only regional characteristic that may matter in this
context. Regions also differ in accumulated experiences and the presence
of tacit knowledge with respect to products, as well as in actors'
propensity share this knowledge among producers (Martin et al., 2019)

To summarize the discussion, first, here the diffusion process is
selfperpetuating as initial use stimulates further use. Second, it
follows a disequilibrium path as the level of users is always lower than
the number of potential users (Stoneman 2002).

The criticisms are mainly based on the fact that although the approach
gives an idea of aggregate (industry or household) behaviour, it does
not focus on the individual's (firm or household) adoption process

{[}8{]} Spatial diffusion is the process by which behavior or
characteristics of the landscape change as a result of what happens
elsewhere earlier. Spatial diffusion is the spread of the phenomenon,
over space and timed, from limited origins

Contagious diffusion=\textgreater{} spatial autocorelation

expansion-type diffusion

The work by the communication sociologist Everett Rogers (1962, 1971,
1983) has emphasized the role of information, communication, formal and
informal media, opinion leaders and social networks, and economic and
psychological constraints on acceptance. Rogers's work stresses the
decision mechanism of the potential adopter; the work of Rogers
partitions the adopter's process of choosing to accept a new phenomenon
or trait into five stages: stage 1--the potential adopter gains
knowledge or awareness of an innovation; stage 2--persuasion is
exercised to adopt; stage 3--a decision is made to adopt; stage 4--the
decision to adopt is implemented; stage 5--the adopter confirms the
decision to adopt. In Rogers's stage theory there may be a substantial
time lag between when a potential adopter becomes aware of the new
characteristic and when a decision is actually made to adopt.

Hägerstrand considered diffusion to be a fundamental geographic process:
Whatever the phenomenon being diffused might be, one may consider it in
the context of a larger universal process of spatial diffusion.

{[}9{]} Diffusion is a spatial (Hagerstrand, 1967) as well as a temporal
process, and historical evidence confirms that technologies diffuse at
different times, at different rates, and to different extents in
different places, and can be significantly influenced by policies
(Victor, 1993).

Lastly, it can be globally optimal for innovative economies to deploy
advanced technologies more than what is locally optimal if this enables
faster diffusion in less advanced regions that could benefit from the
technology.

wilson201281 Early on in their lifecycle, new technologies are crude,
imperfect, and expensive (Rosenberg, 1994). New energy technologies are
attractive for their ability to perform a particular task or deliver a
new or improved energy service (Fouquet, 2010). This is often
circumscribed by a particular set of needs in a particular context: a
market `niche'. End-users in niche markets are generally less sensitive
to the effective price of the energy service provided or have a higher
willingness to pay for its performance advantages (Fouquet, 2010). Thus
initially, performance dominates cost competitiveness (Wilson and
Grubler, 2011). Market niches afford some protection from competitive
pressures, allowing technologies to be tested and improved in applied
settings, reducing uncertainties with performance or market demand (Kemp
et al., 1998). Costs may only fall substantively after an extended
period of commercial experimentation, concurrent with the establishment
of an industrial base and characteristic moves towards standardisation
and mass production (Grubler, 1998). The influence of accumulating
production experience on costs is captured by the concept of learning.

grubler1990rise The scale of spatial innovation diffusion is similar to
that of temporal diffusion models: hierarchically decomposed.

{[}10{]} Later Hagerstrand conceptualized physical ``barrier'' effects
like lakes or uninhabited areas, which, in addition to distance, act as
further retarding effects on diffusion. These are formalized in the form
of ``zero'' or ``half'' contact multiplicators on the (distance
decaying) message flows.

{[}10{]} With respect to the formalization of the communication flows
Hagerstrand defines a ``mean information field'' (MIF), in which the
probability of communication is a negative function of distance between
individuals

{[}2{]} Theoretical models disagree as to why firms adopt innovations at
different times. Epidemic models emphasize information (Griliches 1957).
Certain firms are hypothesized to adopt earlier because they come into
contact with, and learn from, adopters of the new technology before
others. Economic models, on the other hand, predominantly emphasize firm
heterogeneity (Ireland and Stoneman 1986). Firms adopt technologies at
different times because they differ with respect to various
organizational and environmental variables influencing the economic
returns from adoption (Blackman 1999).

Either way, the strong assumption is that developing countries can
acquire modern technology innovated in developed economies, often at a
fraction of the original research and development (R\&D) costs, thereby
leapfrogging many decades of technological progress (Teece 2000).

The first involves a country's geographical location. Recent empirical
work suggests that diffusion is `'geographically localized'' (Globerman,
Kokko, and Sjo¨holm 2000; Keller 2002; Milner 2003) in that a technology
diffuses faster in a country where it is already more widely diffused in
neighboring countries. Underlying these regional effects are contagion
and contact with prior users or producers of technology.

{[}11{]} Rogers (1995) characterized early adopters as knowledgeable
risk takers. Griliches (1957) looked at the rate of return for early
adopters and characterized them as profit maximizers. The geographer
Hagerstrand (1967) was focusing on the character of early adopters in
the transmittal process that produces follow-on participants in the
process. Space was treated as a contiguity system with embedded
characteristics producing barrier elements to the diffusion process.

{[}11{]} REVIEW ON Technological Diffusion

{[}12{]} Two main mechanisms are identified in the literature to explain
diffusion: (1) epidemic-type dynamics whereby contact with previous
adopters stimulates uptake as po tential adopters learn about a new
innovation; and (2) economic-type mechanisms whereby potential users
adopt a new innovation as it be comesmore profitable, useful, or
valuable, with uptake characteristically spreading as costs be come
lower, performance improves, or the po tential uses of the innovation
grow over time.

To this extent, our results contribute to a growing body of work that
has sought to caution against claims about the supposed novelty of the
Internet and the suggestion that it is somehow different.

{[}13{]} Logistic growth describes an initial period of gradual
diffusion as a technology is introduced as a new commercial application,
moving then through a rapid, exponential growth phase, before slowing
and eventually saturating {[}14{]}. The substitution of incumbent
technologies by new competitors leads to subsequent decline and eventual
obsolescence.

\hypertarget{literature-review-sec2-maybe-skip-this}{%
\section{Literature review \{\#sec2\} MAYBE SKIP
THIS??}\label{literature-review-sec2-maybe-skip-this}}

{[}15{]} To characterize evolution of the computer industry, we examine
the distribution of relative employment across cities in 1977 and how
that distribution changes over time to 1992

There is no tendency of relative size distributions of urban computer
employment to collapse, go bimodal, or fully spread. Overall computers
exhibit some turbulence, with dramatic big winners and losers among
cities. In attracting or repelling an industry, urban heterogeneity is
important. Large, well educated cities near San Jose have much greater
chances of attracting high-tech employment and less of losing it

{[}7{]} spatial innovation diffusion By using Bayesian survival models
with time-dependent data of wind turbine deployment and firm foundation
for 402 German regions between the years 1970 and 2015, we show that the
spatial evolution of the German wind energy industry was more strongly
influenced by local demand-- pull than local supply--push processes.

The industry's initial locations are distributed relatively arbitrarily
and unpredictably, as their needs in terms of resources and skills are
diverse and distinct from the older existing industries (Boschma and
Lambooy, 1999). Consequently, emerging industries are characterized by
relatively high degrees of freedom in terms of location. In later
extensions of the concept, the assumption of the randomness of locations
was revised with greater importance assigned to regional conditions
(Boschma and Lambooy, 1999; Fornahl et al., 2012).

{[}2{]} analyze whether the rate at which new producer technologies
diffuse is significantly influenced by (1) latecomer advantage and (2)
engagement with the global economy via trade and foreign investment

Indeed, precisely because of these latecomer advantages, developing
countries1 are believed to be well placed to catch up with developed
ones (Gerschenkron 1962; Abramovitz 1986).

{[}5{]} analyze the spatial diffusion of laser technology research in
West Germany from 1960, when this technology began, until 2005.

{[}12{]}

{[}16{]} We see that the OSN spread almost exclusively from the original
location (the capital Budapest, with an order of magnitude more
inhabitants than the next size town) to various parts of the country in
the early phase of the life-cycle. Later, diffusion became less
mono-centric and other towns also emerged as spreaders. Our findings
support the idea that spreading initially happens to large distances and
becomes more local over time. This is illustrated and discussed later in
Figure 2I.

{[}17{]}

{[}9{]} for energy There is no generally accepted theory that explains
diffusion rate heterogeneity across technologies, but several factors
are considered important. Greater unit scale and larger market size
contribute to slower diffusion. Requirements for interrelated
technologies or complex infrastructures also hinder the diffusion
process (Grubler, 2012).

Mobile phones benefited from early deployment in recreational boats and
automobiles, where the traditional competitor was not a viable option.
In the early stages of diffusion, performance is a more important driver
of adoption than cost competitiveness. Typically, significant cost
reductions only occur once the technology reaches a deployment level
capable of supporting standardization and mass production (Wilson,
2012).

{[}9{]} Empirical evidence supports the validity of Schmidt's Law over a
wide range of technologies, time periods, and geographical contexts. A
recent meta-analysis of technology up-scaling found that diffusion
accelerated moving from the core to the rim and periphery for
technologies as diverse as natural gas power, oil refineries, and
automobiles (Wilson, 2009). One historical example that conforms
particularly well to Schmidt's Law is the diffusion of coal power in
Europe (Grubler, 2012). England emerged as the core region for coal
power because it had legal and economic institutions that incentivized
scientific pursuits, domestic coal reserves, and a clear industrial
motivation to replace water power with coal. \ldots{}

{[}11{]} model the spatial diffusion of mobile telecommunications in
China as well as its determinants. regional socioeconomic
characteristics play an important role in determining the timing, speed,
and level of mobile telecommunications diffusion in China

{[}18{]} explore What determines the duration of formative phases for
energy innovations in different markets? We are interested both in
initial markets (also: core, lead, first mover, early adopter) where
formative phases prepare technologies for mass commercialization, and in
follower markets (also: periphery, lag, late adopter) where accelerated
formative phases may benefit from diffusion and spillovers.

\hypertarget{sec3}, see also {[}10{]}

The literature usually uses the saturation level as the asymptote. I am
using the total number of websites as we cannot compute a rate.

Moran's I as {[}11{]} \textbf{TODO} for t\_0, diffusion speed

\hypertarget{sec3.1}{%
\subsection{Results}\label{sec3.1}}

Hexagon density maps: reflect the spatial structure of Britain. Websites
are associated in places where people live and work.

\textbf{TODO} maps at the local authority level per firms and discuss
patterns

Neighbourhood effect: diffusion proceeds outwards from innovation
centers, first ``hitting'' nearby rather than far-away locations
(Grubler 1990)

\begin{itemize}
\item
  Moran's I: for OA and LAD over time \textbf{TODO} add 0s
\item
  LISA maps: for OA and LAD over time More and less expected clusters.
  Different scales show different results
\item
  Website density regressions: for OA and LAD over time \textbf{TODO}
  add 0s Similar pattern
\end{itemize}

Hierarchy effect: from main centers to secondary ones -- central places

\begin{itemize}
\tightlist
\item
  Gini coefficient. Almost perfect polarisation of web adoption in the
  early stages at a granular level More equally diffused at the Local
  Authority level Plateau overtime
\end{itemize}

S-shaped diffusion curves

\begin{itemize}
\item
  S for LADs per firm and OA. \textbf{TODO}: OA per firm? fix firms over
  time.
\item
  Fast and slow LAs map. There is clustering
\item
  \textbf{TODo} check s\_uk\_firm\_11
\item
  ranks: there is stability and movement
\end{itemize}

\hypertarget{sec3.2}{%
\subsection{3.2 Case study approach}\label{sec3.2}}

\hypertarget{sec4}{%
\section{Digital technologies and spatial structure: a global
view}\label{sec4}}

\hypertarget{sec5}{%
\section{The impact of ICT on the US and the UK spatial
structures}\label{sec5}}

\hypertarget{sec6}{%
\section{Discussion and conclusions}\label{sec6}}

contrary to results from future studies regarding social media {[}16{]},
web technologies did not exclusively spread from a central location.

\hypertarget{references}{%
\section*{References}\label{references}}
\addcontentsline{toc}{section}{References}

\hypertarget{refs}{}
\begin{CSLReferences}{0}{0}
\leavevmode\vadjust pre{\hypertarget{ref-hagerstrand1968innovation}{}}%
\CSLLeftMargin{1. }%
\CSLRightInline{Hagerstrand T et al. Innovation diffusion as a spatial
process. Innovation diffusion as a spatial process. Chicago, USA: Univ.
Chicago Press.; 1968; }

\leavevmode\vadjust pre{\hypertarget{ref-perkins2005international}{}}%
\CSLLeftMargin{2. }%
\CSLRightInline{Perkins R, Neumayer E. The international diffusion of
new technologies: A multitechnology analysis of latecomer advantage and
global economic integration. Annals of the Association of American
Geographers. Taylor \& Francis; 2005;95: 789--808. }

\leavevmode\vadjust pre{\hypertarget{ref-zook2022mapping}{}}%
\CSLLeftMargin{3. }%
\CSLRightInline{Zook M, McCanless M. Mapping the uneven geographies of
digital phenomena: The case of blockchain. The Canadian Geographer/Le
G{é}ographe canadien. Wiley Online Library; 2022;66: 23--36. }

\leavevmode\vadjust pre{\hypertarget{ref-meade2021modelling}{}}%
\CSLLeftMargin{4. }%
\CSLRightInline{Meade N, Islam T. Modelling and forecasting national
introduction times for successive generations of mobile telephony.
Telecommunications Policy. Elsevier; 2021;45: 102088. }

\leavevmode\vadjust pre{\hypertarget{ref-fritsch2015new}{}}%
\CSLLeftMargin{5. }%
\CSLRightInline{Fritsch M, Medrano Echalar LF. New technology in the
region--agglomeration and absorptive capacity effects on laser
technology research in west germany, 1960--2005. Economics of Innovation
and New Technology. Taylor \& Francis; 2015;24: 65--94. }

\leavevmode\vadjust pre{\hypertarget{ref-bento2015spatial}{}}%
\CSLLeftMargin{6. }%
\CSLRightInline{Bento N, Fontes M. Spatial diffusion and the formation
of a technological innovation system in the receiving country: The case
of wind energy in portugal. Environmental Innovation and Societal
Transitions. Elsevier; 2015;15: 158--179. }

\leavevmode\vadjust pre{\hypertarget{ref-bednarz2020pulled}{}}%
\CSLLeftMargin{7. }%
\CSLRightInline{Bednarz M, Broekel T. Pulled or pushed? The spatial
diffusion of wind energy between local demand and supply. Industrial and
Corporate Change. Oxford University Press; 2020;29: 893--916. }

\leavevmode\vadjust pre{\hypertarget{ref-morrill2020spatial}{}}%
\CSLLeftMargin{8. }%
\CSLRightInline{Morrill R, Gaile GL, Thrall GI. Spatial diffusion. Sage
Publications; 1988; }

\leavevmode\vadjust pre{\hypertarget{ref-leibowicz2016representing}{}}%
\CSLLeftMargin{9. }%
\CSLRightInline{Leibowicz BD, Krey V, Grubler A. Representing spatial
technology diffusion in an energy system optimization model.
Technological Forecasting and Social Change. Elsevier; 2016;103:
350--363. }

\leavevmode\vadjust pre{\hypertarget{ref-grubler1990rise}{}}%
\CSLLeftMargin{10. }%
\CSLRightInline{Grubler A. The rise and fall of infrastructures:
Dynamics of evolution and technological change in transport.
Physica-Verlag; 1990. }

\leavevmode\vadjust pre{\hypertarget{ref-ding2010modeling}{}}%
\CSLLeftMargin{11. }%
\CSLRightInline{Ding L, Haynes KE, Li H. Modeling the spatial diffusion
of mobile telecommunications in china. The Professional Geographer.
Taylor \& Francis; 2010;62: 248--263. }

\leavevmode\vadjust pre{\hypertarget{ref-perkins2011internet}{}}%
\CSLLeftMargin{12. }%
\CSLRightInline{Perkins R, Neumayer E. Is the internet really new after
all? The determinants of telecommunications diffusion in historical
perspective. The Professional Geographer. Taylor \& Francis; 2011;63:
55--72. }

\leavevmode\vadjust pre{\hypertarget{ref-wilson201281}{}}%
\CSLLeftMargin{13. }%
\CSLRightInline{Wilson C. Up-scaling, formative phases, and learning in
the historical diffusion of energy technologies. Energy Policy. 2012;50:
81--94. doi:\url{https://doi.org/10.1016/j.enpol.2012.04.077}}

\leavevmode\vadjust pre{\hypertarget{ref-grubler1999dynamics}{}}%
\CSLLeftMargin{14. }%
\CSLRightInline{Grübler A, Nakićenović N, Victor DG. Dynamics of energy
technologies and global change. Energy policy. Elsevier; 1999;27:
247--280. }

\leavevmode\vadjust pre{\hypertarget{ref-beardsell1999spatial}{}}%
\CSLLeftMargin{15. }%
\CSLRightInline{Beardsell M, Henderson V. Spatial evolution of the
computer industry in the USA. European Economic Review. Elsevier;
1999;43: 431--456. }

\leavevmode\vadjust pre{\hypertarget{ref-lengyel2020role}{}}%
\CSLLeftMargin{16. }%
\CSLRightInline{Lengyel B, Bokányi E, Di Clemente R, Kertész J, González
MC. The role of geography in the complex diffusion of innovations.
Scientific reports. Nature Publishing Group; 2020;10: 1--11. }

\leavevmode\vadjust pre{\hypertarget{ref-PAPAGIANNIDIS2015308}{}}%
\CSLLeftMargin{17. }%
\CSLRightInline{Papagiannidis S, Gebka B, Gertner D, Stahl F. Diffusion
of web technologies and practices: A longitudinal study. Technological
Forecasting and Social Change. 2015;96: 308--321.
doi:\url{https://doi.org/10.1016/j.techfore.2015.04.011}}

\leavevmode\vadjust pre{\hypertarget{ref-bento2018time}{}}%
\CSLLeftMargin{18. }%
\CSLRightInline{Bento N, Wilson C, Anadon LD. Time to get ready:
Conceptualizing the temporal and spatial dynamics of formative phases
for energy technologies. Energy Policy. Elsevier; 2018;119: 282--293. }

\end{CSLReferences}

\nolinenumbers



\end{document}
