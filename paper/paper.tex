% Template for PLoS
% Version 3.5 March 2018
%
% % % % % % % % % % % % % % % % % % % % % %
%
% -- IMPORTANT NOTE
%
% This template contains comments intended
% to minimize problems and delays during our production
% process. Please follow the template instructions
% whenever possible.
%
% % % % % % % % % % % % % % % % % % % % % % %
%
% Once your paper is accepted for publication,
% PLEASE REMOVE ALL TRACKED CHANGES in this file
% and leave only the final text of your manuscript.
% PLOS recommends the use of latexdiff to track changes during review, as this will help to maintain a clean tex file.
% Visit https://www.ctan.org/pkg/latexdiff?lang=en for info or contact us at latex@plos.org.
%
%
% There are no restrictions on package use within the LaTeX files except that
% no packages listed in the template may be deleted.
%
% Please do not include colors or graphics in the text.
%
% The manuscript LaTeX source should be contained within a single file (do not use \input, \externaldocument, or similar commands).
%
% % % % % % % % % % % % % % % % % % % % % % %
%
% -- FIGURES AND TABLES
%
% Please include tables/figure captions directly after the paragraph where they are first cited in the text.
%
% DO NOT INCLUDE GRAPHICS IN YOUR MANUSCRIPT
% - Figures should be uploaded separately from your manuscript file.
% - Figures generated using LaTeX should be extracted and removed from the PDF before submission.
% - Figures containing multiple panels/subfigures must be combined into one image file before submission.
% For figure citations, please use "Fig" instead of "Figure".
% See http://journals.plos.org/plosone/s/figures for PLOS figure guidelines.
%
% Tables should be cell-based and may not contain:
% - spacing/line breaks within cells to alter layout or alignment
% - do not nest tabular environments (no tabular environments within tabular environments)
% - no graphics or colored text (cell background color/shading OK)
% See http://journals.plos.org/plosone/s/tables for table guidelines.
%
% For tables that exceed the width of the text column, use the adjustwidth environment as illustrated in the example table in text below.
%
% % % % % % % % % % % % % % % % % % % % % % % %
%
% -- EQUATIONS, MATH SYMBOLS, SUBSCRIPTS, AND SUPERSCRIPTS
%
% IMPORTANT
% Below are a few tips to help format your equations and other special characters according to our specifications. For more tips to help reduce the possibility of formatting errors during conversion, please see our LaTeX guidelines at http://journals.plos.org/plosone/s/latex
%
% For inline equations, please be sure to include all portions of an equation in the math environment.
%
% Do not include text that is not math in the math environment.
%
% Please add line breaks to long display equations when possible in order to fit size of the column.
%
% For inline equations, please do not include punctuation (commas, etc) within the math environment unless this is part of the equation.
%
% When adding superscript or subscripts outside of brackets/braces, please group using {}.
%
% Do not use \cal for caligraphic font.  Instead, use \mathcal{}
%
% % % % % % % % % % % % % % % % % % % % % % % %
%
% Please contact latex@plos.org with any questions.
%
% % % % % % % % % % % % % % % % % % % % % % % %

\documentclass[10pt,letterpaper]{article}
\usepackage[top=0.85in,left=2.75in,footskip=0.75in]{geometry}

% amsmath and amssymb packages, useful for mathematical formulas and symbols
\usepackage{amsmath,amssymb}

% Use adjustwidth environment to exceed column width (see example table in text)
\usepackage{changepage}

% Use Unicode characters when possible
\usepackage[utf8x]{inputenc}

% textcomp package and marvosym package for additional characters
\usepackage{textcomp,marvosym}

% cite package, to clean up citations in the main text. Do not remove.
% \usepackage{cite}

% Use nameref to cite supporting information files (see Supporting Information section for more info)
\usepackage{nameref,hyperref}

% line numbers
\usepackage[right]{lineno}

% ligatures disabled
\usepackage{microtype}
\DisableLigatures[f]{encoding = *, family = * }

% color can be used to apply background shading to table cells only
\usepackage[table]{xcolor}

% array package and thick rules for tables
\usepackage{array}

% create "+" rule type for thick vertical lines
\newcolumntype{+}{!{\vrule width 2pt}}

% create \thickcline for thick horizontal lines of variable length
\newlength\savedwidth
\newcommand\thickcline[1]{%
  \noalign{\global\savedwidth\arrayrulewidth\global\arrayrulewidth 2pt}%
  \cline{#1}%
  \noalign{\vskip\arrayrulewidth}%
  \noalign{\global\arrayrulewidth\savedwidth}%
}

% \thickhline command for thick horizontal lines that span the table
\newcommand\thickhline{\noalign{\global\savedwidth\arrayrulewidth\global\arrayrulewidth 2pt}%
\hline
\noalign{\global\arrayrulewidth\savedwidth}}


% Remove comment for double spacing
%\usepackage{setspace}
%\doublespacing

% Text layout
\raggedright
\setlength{\parindent}{0.5cm}
\textwidth 5.25in
\textheight 8.75in

% Bold the 'Figure #' in the caption and separate it from the title/caption with a period
% Captions will be left justified
\usepackage[aboveskip=1pt,labelfont=bf,labelsep=period,justification=raggedright,singlelinecheck=off]{caption}
\renewcommand{\figurename}{Fig}

% Use the PLoS provided BiBTeX style
% \bibliographystyle{plos2015}

% Remove brackets from numbering in List of References
\makeatletter
\renewcommand{\@biblabel}[1]{\quad#1.}
\makeatother



% Header and Footer with logo
\usepackage{lastpage,fancyhdr,graphicx}
\usepackage{epstopdf}
%\pagestyle{myheadings}
\pagestyle{fancy}
\fancyhf{}
%\setlength{\headheight}{27.023pt}
%\lhead{\includegraphics[width=2.0in]{PLOS-submission.eps}}
\rfoot{\thepage/\pageref{LastPage}}
\renewcommand{\headrulewidth}{0pt}
\renewcommand{\footrule}{\hrule height 2pt \vspace{2mm}}
\fancyheadoffset[L]{2.25in}
\fancyfootoffset[L]{2.25in}
\lfoot{\today}

%% Include all macros below

\newcommand{\lorem}{{\bf LOREM}}
\newcommand{\ipsum}{{\bf IPSUM}}


% Pandoc citation processing
\newlength{\csllabelwidth}
\setlength{\csllabelwidth}{3em}
\newlength{\cslhangindent}
\setlength{\cslhangindent}{1.5em}
% for Pandoc 2.8 to 2.10.1
\newenvironment{cslreferences}%
  {}%
  {\par}
% For Pandoc 2.11+
\newenvironment{CSLReferences}[2] % #1 hanging-ident, #2 entry spacing
 {% don't indent paragraphs
  \setlength{\parindent}{0pt}
  % turn on hanging indent if param 1 is 1
  \ifodd #1 \everypar{\setlength{\hangindent}{\cslhangindent}}\ignorespaces\fi
  % set entry spacing
  \ifnum #2 > 0
  \setlength{\parskip}{#2\baselineskip}
  \fi
 }%
 {}
\usepackage{calc} % for calculating minipage widths
\newcommand{\CSLBlock}[1]{#1\hfill\break}
\newcommand{\CSLLeftMargin}[1]{\parbox[t]{\csllabelwidth}{#1}}
\newcommand{\CSLRightInline}[1]{\parbox[t]{\linewidth - \csllabelwidth}{#1}\break}
\newcommand{\CSLIndent}[1]{\hspace{\cslhangindent}#1}

\usepackage{rotating, graphicx, dcolumn, booktabs, caption, xcolor}



\usepackage{forarray}
\usepackage{xstring}
\newcommand{\getIndex}[2]{
  \ForEach{,}{\IfEq{#1}{\thislevelitem}{\number\thislevelcount\ExitForEach}{}}{#2}
}

\setcounter{secnumdepth}{0}

\newcommand{\getAff}[1]{
  \getIndex{#1}{University of Bristol and The Alan Turing Institute}
}

\providecommand{\tightlist}{%
  \setlength{\itemsep}{0pt}\setlength{\parskip}{0pt}}

\begin{document}
\vspace*{0.2in}

% Title must be 250 characters or less.
\begin{flushleft}
{\Large
\textbf\newline{Title of submission to PLOS
journal} % Please use "sentence case" for title and headings (capitalize only the first word in a title (or heading), the first word in a subtitle (or subheading), and any proper nouns).
}
\newline
% Insert author names, affiliations and corresponding author email (do not include titles, positions, or degrees).
\\
Emmanouil Tranos\textsuperscript{\getAff{University of Bristol and The
Alan Turing Institute}}\textsuperscript{*}\\
\bigskip
\textbf{\getAff{University of Bristol and The Alan Turing
Institute}}University of Bristol, School of Geographical Sciences,
Bristol, UK; The Alan Turing Institute, London, UK\\
\bigskip
* Corresponding author: e.tranos@bristol.ac.uk\\
\end{flushleft}
% Please keep the abstract below 300 words
\section*{Abstract}
blah blah

% Please keep the Author Summary between 150 and 200 words
% Use first person. PLOS ONE authors please skip this step.
% Author Summary not valid for PLOS ONE submissions.
\section*{Author summary}
blah blah

\linenumbers

% Use "Eq" instead of "Equation" for equation citations.
\hypertarget{sec:1}{%
\section{Introduction}\label{sec:1}}

{[}1{]} To characterize evolution of the computer industry, we examine
the distribution of relative employment across cities in 1977 and how
that distribution changes over time to 1992

There is no tendency of relative size distributions of urban computer
employment to collapse, go bimodal, or fully spread. Overall computers
exhibit some turbulence, with dramatic big winners and losers among
cities. In attracting or repelling an industry, urban heterogeneity is
important. Large, well educated cities near San Jose have much greater
chances of attracting high-tech employment and less of losing it

{[}2{]} spatial innovation diffusion By using Bayesian survival models
with time-dependent data of wind turbine deployment and firm foundation
for 402 German regions between the years 1970 and 2015, we show that the
spatial evolution of the German wind energy industry was more strongly
influenced by local demand-- pull than local supply--push processes.

The industry's initial locations are distributed relatively arbitrarily
and unpredictably, as their needs in terms of resources and skills are
diverse and distinct from the older existing industries (Boschma and
Lambooy, 1999). Consequently, emerging industries are characterized by
relatively high degrees of freedom in terms of location. In later
extensions of the concept, the assumption of the randomness of locations
was revised with greater importance assigned to regional conditions
(Boschma and Lambooy, 1999; Fornahl et al., 2012).

larger local technological niches (at least from a demand perspective)
are more likely to be formed in urban areas than in rural ones, which is
in line with the work of Ha¨gerstrand (1965a,b). However, size is not
the only regional characteristic that may matter in this context.
Regions also differ in accumulated experiences and the presence of tacit
knowledge with respect to products, as well as in actors' propensity
share this knowledge among producers (Martin et al., 2019)

{[}3{]} The concept of absorptive capacity has also been applied in the
TIS context in terms of the con-struction of capacity needed to transfer
and implement a new technology in the receiving country (cf.Van Alphen,
2011), with repercussions for the nature of system functions

{[}4not good paper{]} Invention, innovation and diffusion are considered
as the three pillars of the theory of technological change and
Schumpeter is considered as the father of the study of diffusion

Epidemic approach: The approach emphasizes that the adoption of a
technology spreads like an infection among the potential adopters. By
taking a homogeneous group of potential adopters as an assumption, it
explains the adoption of any innovation.

To summarize the discussion, first, here the diffusion process is
selfperpetuating as initial use stimulates further use. Second, it
follows a disequilibrium path as the level of users is always lower than
the number of potential users (Stoneman 2002).

The criticisms are mainly based on the fact that although the approach
gives an idea of aggregate (industry or household) behaviour, it does
not focus on the individual's (firm or household) adoption process

Rogers3, that divides adopters as: (I.) Innovators: first 2.5\%, (II.)
Early adopters: next 13.5\%, (III.) Early Majority: following 34\%,
(IV.) Late Majority: next 34\%, and the rest as (V.) Laggards

{[}5{]} Spatial diffusion is the process by which behavior or
characteristics of the landscape change as a result of what happens
elsewhere earlier. Spatial diffusion is the spread of the phenomenon,
over space and timed, from limited origins

Contagious diffusion=\textgreater{} spatial autocorelation

expansion-type diffusion

the three geographic diffusion processes are (1) purely contagious where
distance or adjacency is the absolute controlling factor, (2) purely
hierarchical where size or urban position in the central place hierarchy
is the absolute controlling factor, or (3) where the location of change
is purely random. These are the pure or ideal forms of the diffusion
process; all real diffusion processes are the result of a combination of
these extremes.

The work by the communication sociologist Everett Rogers (1962, 1971,
1983) has emphasized the role of information, communication, formal and
informal media, opinion leaders and social networks, and economic and
psychological constraints on acceptance. Rogers's work stresses the
decision mechanism of the potential adopter; the work of Rogers
partitions the adopter's process of choosing to accept a new phenomenon
or trait into five stages: stage 1--the potential adopter gains
knowledge or awareness of an innovation; stage 2--persuasion is
exercised to adopt; stage 3--a decision is made to adopt; stage 4--the
decision to adopt is implemented; stage 5--the adopter confirms the
decision to adopt. In Rogers's stage theory there may be a substantial
time lag between when a potential adopter becomes aware of the new
characteristic and when a decision is actually made to adopt.

Hägerstrand considered diffusion to be a fundamental geographic process:
Whatever the phenomenon being diffused might be, one may consider it in
the context of a larger universal process of spatial diffusion.

{[}6{]} Because new digital activities are rarely---if ever---captured
in official state data, researchers must rely on information gathered
from alternative sources. With this in mind, the overarching aim of this
article is outlining an approach for analyzing the ``new spaces and
geographies'' of digital phenomena and practices.

\hypertarget{literature-review-sec2-maybe-skip-this}{%
\section{Literature review \{\#sec2\} MAYBE SKIP
THIS??}\label{literature-review-sec2-maybe-skip-this}}

\hypertarget{sec3}{%
\section{Materials and Methods}\label{sec3}}

\hypertarget{sec3.1}{%
\subsection{Multi-country analysis}\label{sec3.1}}

\hypertarget{sec3.2}{%
\subsection{3.2 Case study approach}\label{sec3.2}}

\hypertarget{sec4}{%
\section{Digital technologies and spatial structure: a global
view}\label{sec4}}

\hypertarget{sec5}{%
\section{The impact of ICT on the US and the UK spatial
structures}\label{sec5}}

\hypertarget{sec6}{%
\section{Discussion and conclusions}\label{sec6}}

contrary to results from future studies regarding social media {[}7{]},
web technologies did not exclusively spread from a central location.

{[}7{]} We see that the OSN spread almost exclusively from the original
location (the capital Budapest, with an order of magnitude more
inhabitants than the next size town) to various parts of the country in
the early phase of the life-cycle. Later, diffusion became less
mono-centric and other towns also emerged as spreaders. Our findings
support the idea that spreading initially happens to large distances and
becomes more local over time. This is illustrated and discussed later in
Figure 2I.

\hypertarget{references}{%
\section*{References}\label{references}}
\addcontentsline{toc}{section}{References}

\hypertarget{refs}{}
\begin{CSLReferences}{0}{0}
\leavevmode\hypertarget{ref-beardsell1999spatial}{}%
\CSLLeftMargin{1. }
\CSLRightInline{Beardsell M, Henderson V. Spatial evolution of the
computer industry in the USA. European Economic Review. Elsevier;
1999;43: 431--456. }

\leavevmode\hypertarget{ref-bednarz2020pulled}{}%
\CSLLeftMargin{2. }
\CSLRightInline{Bednarz M, Broekel T. Pulled or pushed? The spatial
diffusion of wind energy between local demand and supply. Industrial and
Corporate Change. Oxford University Press; 2020;29: 893--916. }

\leavevmode\hypertarget{ref-bento2015spatial}{}%
\CSLLeftMargin{3. }
\CSLRightInline{Bento N, Fontes M. Spatial diffusion and the formation
of a technological innovation system in the receiving country: The case
of wind energy in portugal. Environmental Innovation and Societal
Transitions. Elsevier; 2015;15: 158--179. }

\leavevmode\hypertarget{ref-das2022diffusion}{}%
\CSLLeftMargin{4. }
\CSLRightInline{Das B. Diffusion of innovations: Theoretical
perspectives and empirical evidence. African Journal of Science,
Technology, Innovation and Development. Taylor \& Francis; 2022;14:
94--103. }

\leavevmode\hypertarget{ref-morrill2020spatial}{}%
\CSLLeftMargin{5. }
\CSLRightInline{Morrill R, Gaile GL, Thrall GI. Spatial diffusion. Sage
Publications; 1988; }

\leavevmode\hypertarget{ref-zook2022mapping}{}%
\CSLLeftMargin{6. }
\CSLRightInline{Zook M, McCanless M. Mapping the uneven geographies of
digital phenomena: The case of blockchain. The Canadian Geographer/Le
G{é}ographe canadien. Wiley Online Library; 2022;66: 23--36. }

\leavevmode\hypertarget{ref-lengyel2020role}{}%
\CSLLeftMargin{7. }
\CSLRightInline{Lengyel B, Bokányi E, Di Clemente R, Kertész J, González
MC. The role of geography in the complex diffusion of innovations.
Scientific reports. Nature Publishing Group; 2020;10: 1--11. }

\end{CSLReferences}

\nolinenumbers


\end{document}
