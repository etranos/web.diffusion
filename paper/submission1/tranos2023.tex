% Options for packages loaded elsewhere
\PassOptionsToPackage{unicode}{hyperref}
\PassOptionsToPackage{hyphens}{url}
\PassOptionsToPackage{dvipsnames,svgnames,x11names}{xcolor}
%
\documentclass[
  authoryear,
  preprint,
  3p]{elsarticle}

\usepackage{amsmath,amssymb}
\usepackage{lmodern}
\usepackage{iftex}
\ifPDFTeX
  \usepackage[T1]{fontenc}
  \usepackage[utf8]{inputenc}
  \usepackage{textcomp} % provide euro and other symbols
\else % if luatex or xetex
  \usepackage{unicode-math}
  \defaultfontfeatures{Scale=MatchLowercase}
  \defaultfontfeatures[\rmfamily]{Ligatures=TeX,Scale=1}
\fi
% Use upquote if available, for straight quotes in verbatim environments
\IfFileExists{upquote.sty}{\usepackage{upquote}}{}
\IfFileExists{microtype.sty}{% use microtype if available
  \usepackage[]{microtype}
  \UseMicrotypeSet[protrusion]{basicmath} % disable protrusion for tt fonts
}{}
\makeatletter
\@ifundefined{KOMAClassName}{% if non-KOMA class
  \IfFileExists{parskip.sty}{%
    \usepackage{parskip}
  }{% else
    \setlength{\parindent}{0pt}
    \setlength{\parskip}{6pt plus 2pt minus 1pt}}
}{% if KOMA class
  \KOMAoptions{parskip=half}}
\makeatother
\usepackage{xcolor}
\setlength{\emergencystretch}{3em} % prevent overfull lines
\setcounter{secnumdepth}{5}
% Make \paragraph and \subparagraph free-standing
\ifx\paragraph\undefined\else
  \let\oldparagraph\paragraph
  \renewcommand{\paragraph}[1]{\oldparagraph{#1}\mbox{}}
\fi
\ifx\subparagraph\undefined\else
  \let\oldsubparagraph\subparagraph
  \renewcommand{\subparagraph}[1]{\oldsubparagraph{#1}\mbox{}}
\fi


\providecommand{\tightlist}{%
  \setlength{\itemsep}{0pt}\setlength{\parskip}{0pt}}\usepackage{longtable,booktabs,array}
\usepackage{calc} % for calculating minipage widths
% Correct order of tables after \paragraph or \subparagraph
\usepackage{etoolbox}
\makeatletter
\patchcmd\longtable{\par}{\if@noskipsec\mbox{}\fi\par}{}{}
\makeatother
% Allow footnotes in longtable head/foot
\IfFileExists{footnotehyper.sty}{\usepackage{footnotehyper}}{\usepackage{footnote}}
\makesavenoteenv{longtable}
\usepackage{graphicx}
\makeatletter
\def\maxwidth{\ifdim\Gin@nat@width>\linewidth\linewidth\else\Gin@nat@width\fi}
\def\maxheight{\ifdim\Gin@nat@height>\textheight\textheight\else\Gin@nat@height\fi}
\makeatother
% Scale images if necessary, so that they will not overflow the page
% margins by default, and it is still possible to overwrite the defaults
% using explicit options in \includegraphics[width, height, ...]{}
\setkeys{Gin}{width=\maxwidth,height=\maxheight,keepaspectratio}
% Set default figure placement to htbp
\makeatletter
\def\fps@figure{htbp}
\makeatother

\makeatletter
\makeatother
\makeatletter
\makeatother
\makeatletter
\@ifpackageloaded{caption}{}{\usepackage{caption}}
\AtBeginDocument{%
\ifdefined\contentsname
  \renewcommand*\contentsname{Table of contents}
\else
  \newcommand\contentsname{Table of contents}
\fi
\ifdefined\listfigurename
  \renewcommand*\listfigurename{List of Figures}
\else
  \newcommand\listfigurename{List of Figures}
\fi
\ifdefined\listtablename
  \renewcommand*\listtablename{List of Tables}
\else
  \newcommand\listtablename{List of Tables}
\fi
\ifdefined\figurename
  \renewcommand*\figurename{Figure}
\else
  \newcommand\figurename{Figure}
\fi
\ifdefined\tablename
  \renewcommand*\tablename{Table}
\else
  \newcommand\tablename{Table}
\fi
}
\@ifpackageloaded{float}{}{\usepackage{float}}
\floatstyle{ruled}
\@ifundefined{c@chapter}{\newfloat{codelisting}{h}{lop}}{\newfloat{codelisting}{h}{lop}[chapter]}
\floatname{codelisting}{Listing}
\newcommand*\listoflistings{\listof{codelisting}{List of Listings}}
\makeatother
\makeatletter
\@ifpackageloaded{caption}{}{\usepackage{caption}}
\@ifpackageloaded{subcaption}{}{\usepackage{subcaption}}
\makeatother
\makeatletter
\@ifpackageloaded{tcolorbox}{}{\usepackage[many]{tcolorbox}}
\makeatother
\makeatletter
\@ifundefined{shadecolor}{\definecolor{shadecolor}{rgb}{.97, .97, .97}}
\makeatother
\makeatletter
\makeatother
\journal{Journal Name}
\ifLuaTeX
  \usepackage{selnolig}  % disable illegal ligatures
\fi
\usepackage[]{natbib}
\bibliographystyle{elsarticle-harv}
\IfFileExists{bookmark.sty}{\usepackage{bookmark}}{\usepackage{hyperref}}
\IfFileExists{xurl.sty}{\usepackage{xurl}}{} % add URL line breaks if available
\urlstyle{same} % disable monospaced font for URLs
\hypersetup{
  pdftitle={Digital economy in the UK: a multi-scalar story of the diffusion of web technologies},
  pdfauthor={Emmanouil Tranos},
  pdfkeywords={keyword1, keyword2},
  colorlinks=true,
  linkcolor={blue},
  filecolor={Maroon},
  citecolor={Blue},
  urlcolor={Blue},
  pdfcreator={LaTeX via pandoc}}

\setlength{\parindent}{6pt}
\begin{document}

\begin{frontmatter}
\title{Digital economy in the UK: a multi-scalar story of the diffusion
of web technologies}
\author[1]{Emmanouil Tranos%
\corref{cor1}%
}
 \ead{e.tranos@bristol.ac.uk} 

\affiliation[1]{organization={University of Bristol and The Alan Turing
Institute},postcode={UK},postcodesep={}}

\cortext[cor1]{Corresponding author}

        
\begin{abstract}
This paper maps the participation in the digital economy and its
evolution in the UK over space and time. Most of the existing economic
geography literature which dealt with the spatiality of the internet
employed supply-side measures, such as infrastructural capacity, in
order to understand the geography of the digital economy and its
potential spatial economic effects. Useful as these approaches might
have been, they cannot capture the micro-processes and the
characteristics of the individual online behaviour. Using large volumes
of archived and geolocated web content, this paper models the diffusion
of web technologies over space and time in the UK. The data and
geolocation strategy allow to capture these processes at a very granular
spatial scale. The modelling approach, which is based on simple spatial
analytical methods and on the estimation of diffusion curves at various
scales, enables to depict the role of geography and other cognitive
factors which drove the diffusion of web technologies. Although the
focus is on a recent historical period -- 1996-2012 -- the results of
the analysis depict diffusion mechanisms which can be very useful in
understanding the evolutionary patterns of the adoption of other newer
technologies.
\end{abstract}





\begin{keyword}
    keyword1 \sep 
    keyword2
\end{keyword}
\end{frontmatter}
    \ifdefined\Shaded\renewenvironment{Shaded}{\begin{tcolorbox}[sharp corners, boxrule=0pt, interior hidden, borderline west={3pt}{0pt}{shadecolor}, breakable, enhanced, frame hidden]}{\end{tcolorbox}}\fi

\hypertarget{sec1}{%
\section{Introduction}\label{sec1}}

Geographers were always interested in how new technologies and
innovations diffuse across space and, importantly, how such
spatio-temporal processes can be modelled. After all, diffusion together
with invention and innovation are considered the pillars of
technological change \citep{das2022diffusion}. The seminal contribution
of \citet{hagerstrand1968innovation} is illustrative of this early
interest. However, the torch of exploring and modelling such processes
had been passed to other disciplines such as economics, business studies
and sociology well before the `cultural turn' of economic geography
\citep{perkins2005international}. A potential explanation of the lack of
geographical studies exploring the diffusion of new and, more
specifically, digital technologies across \emph{both} space and time can
be attributed to the scarcity of relevant and granular enough data. As
\citet{zook2022mapping} highlight, digital activities are hardly ever
captured in official data.

This paper offers such a contribution: a geographical study illustrating
how a new technology that is the web diffused over space and time in the
UK at a high level of spatial granularity during the \(1996-2012\)
period. It does so by employing a novel source of big data which
captures the active engagement with web technologies during that period.
By addressing this empirical question this paper exemplifies how the
combination of data sources which escape the traditional social science
domain and adequate research methods can offer new contributions
regarding our understanding of technologies are diffused.

The motivation for this paper lies in the fact that there are various
stakeholders who are interested in knowing how new digital technologies
diffused over space and time and use this information to make
predictions regarding the diffusion of related \emph{future}
technologies. As per \citet{leibowicz2016representing}, historical
studies agree on the fact that ``technologies diffuse at different
times, at different rates, and to different extents in different places,
and can be significantly influenced by policies
\citep{victor1993}''.@meade2021modelling highlight that a variety of
actors have a direct interest in gaining such knowledge including
network equipment suppliers; network operators, regulatory and local
authorities. These processes and their effects very a lot between
scales: although the diffusion of a new technology might not be optimal
at a local level, it might be beneficial from a global perspective as it
might enable faster diffusion to less advantaged places
\citet{leibowicz2016representing}. Despite the spatial heterogeneity of
such diffusion mechanisms and the policy relevance, there are very
limited attempts in the literature to analyse the diffusion of new
digital technologies at a detailed geographical level.

To summarize the discussion, first, here the diffusion process is
selfperpetuating as initial use stimulates further use. Second, it
follows a disequilibrium path as the level of users is always lower than
the number of potential users (Stoneman 2002).

The criticisms are mainly based on the fact that although the approach
gives an idea of aggregate (industry or household) behaviour, it does
not focus on the individual's (firm or household) adoption process

\citet{morrill2020spatial} Spatial diffusion is the process by which
behavior or characteristics of the landscape change as a result of what
happens elsewhere earlier. Spatial diffusion is the spread of the
phenomenon, over space and timed, from limited origins

Compare dissusion with adoption

wilson201281 Early on in their lifecycle, new technologies are crude,
imperfect, and expensive (Rosenberg, 1994). New energy technologies are
attractive for their ability to perform a particular task or deliver a
new or improved energy service (Fouquet, 2010). This is often
circumscribed by a particular set of needs in a particular context: a
market `niche'. End-users in niche markets are generally less sensitive
to the effective price of the energy service provided or have a higher
willingness to pay for its performance advantages (Fouquet, 2010). Thus
initially, performance dominates cost competitiveness (Wilson and
Grubler, 2011). Market niches afford some protection from competitive
pressures, allowing technologies to be tested and improved in applied
settings, reducing uncertainties with performance or market demand (Kemp
et al., 1998). Costs may only fall substantively after an extended
period of commercial experimentation, concurrent with the establishment
of an industrial base and characteristic moves towards standardisation
and mass production (Grubler, 1998). The influence of accumulating
production experience on costs is captured by the concept of learning.

grubler1990rise The scale of spatial innovation diffusion is similar to
that of temporal diffusion models: hierarchically decomposed.

\citet{grubler1990rise} Later Hagerstrand conceptualized physical
``barrier'' effects like lakes or uninhabited areas, which, in addition
to distance, act as further retarding effects on diffusion. These are
formalized in the form of ``zero'' or ``half'' contact multiplicators on
the (distance decaying) message flows.

\citet{grubler1990rise} With respect to the formalization of the
communication flows Hagerstrand defines a ``mean information field''
(MIF), in which the probability of communication is a negative function
of distance between individuals

\citet{perkins2005international} Theoretical models disagree as to why
firms adopt innovations at different times. Epidemic models emphasize
information (Griliches 1957). Certain firms are hypothesized to adopt
earlier because they come into contact with, and learn from, adopters of
the new technology before others. Economic models, on the other hand,
predominantly emphasize firm heterogeneity (Ireland and Stoneman 1986).
Firms adopt technologies at different times because they differ with
respect to various organizational and environmental variables
influencing the economic returns from adoption (Blackman 1999).

Either way, the strong assumption is that developing countries can
acquire modern technology innovated in developed economies, often at a
fraction of the original research and development (R\&D) costs, thereby
leapfrogging many decades of technological progress (Teece 2000).

The first involves a country's geographical location. Recent empirical
work suggests that diffusion is `'geographically localized'' (Globerman,
Kokko, and Sjo¨holm 2000; Keller 2002; Milner 2003) in that a technology
diffuses faster in a country where it is already more widely diffused in
neighboring countries. Underlying these regional effects are contagion
and contact with prior users or producers of technology.

\citet{ding2010modeling} Rogers (1995) characterized early adopters as
knowledgeable risk takers. Griliches (1957) looked at the rate of return
for early adopters and characterized them as profit maximizers. The
geographer Hagerstrand (1967) was focusing on the character of early
adopters in the transmittal process that produces follow-on participants
in the process. Space was treated as a contiguity system with embedded
characteristics producing barrier elements to the diffusion process.

\citet{ding2010modeling} REVIEW ON Technological Diffusion

\citet{perkins2011internet} Two main mechanisms are identified in the
literature to explain diffusion: (1) epidemic-type dynamics whereby
contact with previous adopters stimulates uptake as po tential adopters
learn about a new innovation; and (2) economic-type mechanisms whereby
potential users adopt a new innovation as it be comesmore profitable,
useful, or valuable, with uptake characteristically spreading as costs
be come lower, performance improves, or the po tential uses of the
innovation grow over time.

To this extent, our results contribute to a growing body of work that
has sought to caution against claims about the supposed novelty of the
Internet and the suggestion that it is somehow different.

\citet{wilson201281} Logistic growth describes an initial period of
gradual diffusion as a technology is introduced as a new commercial
application, moving then through a rapid, exponential growth phase,
before slowing and eventually saturating \citep{grubler1999dynamics}.
The substitution of incumbent technologies by new competitors leads to
subsequent decline and eventual obsolescence.

\hypertarget{sec2}{%
\section{Literature review}\label{sec2}}

MAYBE SKIP THIS??

\citet{beardsell1999spatial} To characterize evolution of the computer
industry, we examine the distribution of relative employment across
cities in 1977 and how that distribution changes over time to 1992

There is no tendency of relative size distributions of urban computer
employment to collapse, go bimodal, or fully spread. Overall computers
exhibit some turbulence, with dramatic big winners and losers among
cities. In attracting or repelling an industry, urban heterogeneity is
important. Large, well educated cities near San Jose have much greater
chances of attracting high-tech employment and less of losing it

\citet{bednarz2020pulled} spatial innovation diffusion By using Bayesian
survival models with time-dependent data of wind turbine deployment and
firm foundation for 402 German regions between the years 1970 and 2015,
we show that the spatial evolution of the German wind energy industry
was more strongly influenced by local demand-- pull than local
supply--push processes.

The industry's initial locations are distributed relatively arbitrarily
and unpredictably, as their needs in terms of resources and skills are
diverse and distinct from the older existing industries (Boschma and
Lambooy, 1999). Consequently, emerging industries are characterized by
relatively high degrees of freedom in terms of location. In later
extensions of the concept, the assumption of the randomness of locations
was revised with greater importance assigned to regional conditions
(Boschma and Lambooy, 1999; Fornahl et al., 2012).

\citet{perkins2005international} analyze whether the rate at which new
producer technologies diffuse is significantly influenced by (1)
latecomer advantage and (2) engagement with the global economy via trade
and foreign investment

Indeed, precisely because of these latecomer advantages, developing
countries1 are believed to be well placed to catch up with developed
ones (Gerschenkron 1962; Abramovitz 1986).

\citet{fritsch2015new} analyze the spatial diffusion of laser technology
research in West Germany from 1960, when this technology began, until
2005.

\citet{perkins2011internet}

\citet{lengyel2020role} We see that the OSN spread almost exclusively
from the original location (the capital Budapest, with an order of
magnitude more inhabitants than the next size town) to various parts of
the country in the early phase of the life-cycle. Later, diffusion
became less mono-centric and other towns also emerged as spreaders. Our
findings support the idea that spreading initially happens to large
distances and becomes more local over time. This is illustrated and
discussed later in Figure 2I.

\citet{PAPAGIANNIDIS2015308}

\citet{leibowicz2016representing} for energy There is no generally
accepted theory that explains diffusion rate heterogeneity across
technologies, but several factors are considered important. Greater unit
scale and larger market size contribute to slower diffusion.
Requirements for interrelated technologies or complex infrastructures
also hinder the diffusion process (Grubler, 2012).

Mobile phones benefited from early deployment in recreational boats and
automobiles, where the traditional competitor was not a viable option.
In the early stages of diffusion, performance is a more important driver
of adoption than cost competitiveness. Typically, significant cost
reductions only occur once the technology reaches a deployment level
capable of supporting standardization and mass production (Wilson,
2012).

\citet{leibowicz2016representing} Empirical evidence supports the
validity of Schmidt's Law over a wide range of technologies, time
periods, and geographical contexts. A recent meta-analysis of technology
up-scaling found that diffusion accelerated moving from the core to the
rim and periphery for technologies as diverse as natural gas power, oil
refineries, and automobiles (Wilson, 2009). One historical example that
conforms particularly well to Schmidt's Law is the diffusion of coal
power in Europe (Grubler, 2012). England emerged as the core region for
coal power because it had legal and economic institutions that
incentivized scientific pursuits, domestic coal reserves, and a clear
industrial motivation to replace water power with coal. \ldots{}

\citet{ding2010modeling} model the spatial diffusion of mobile
telecommunications in China as well as its determinants. regional
socioeconomic characteristics play an important role in determining the
timing, speed, and level of mobile telecommunications diffusion in China

\citet{bento2018time} explore What determines the duration of formative
phases for energy innovations in different markets? We are interested
both in initial markets (also: core, lead, first mover, early adopter)
where formative phases prepare technologies for mass commercialization,
and in follower markets (also: periphery, lag, late adopter) where
accelerated formative phases may benefit from diffusion and spillovers.

\hypertarget{sec3}, see also \citet{grubler1990rise}

The literature usually uses the saturation level as the asymptote. I am
using the total number of websites as we cannot compute a rate.

Moran's I as \citet{ding2010modeling} \textbf{TODO} for t\_0, diffusion
speed

\hypertarget{sec3.1}{%
\subsection{Results}\label{sec3.1}}

Hexagon density maps: reflect the spatial structure of Britain. Websites
are associated in places where people live and work.

\textbf{TODO} maps at the local authority level per firms and discuss
patterns

Neighbourhood effect: diffusion proceeds outwards from innovation
centers, first ``hitting'' nearby rather than far-away locations
(Grubler 1990)

\begin{itemize}
\item
  Moran's I: for OA and LAD over time \textbf{TODO} add 0s
\item
  LISA maps: for OA and LAD over time More and less expected clusters.
  Different scales show different results
\item
  Website density regressions: for OA and LAD over time \textbf{TODO}
  add 0s Similar pattern
\end{itemize}

Hierarchy effect: from main centers to secondary ones -- central places

\begin{itemize}
\tightlist
\item
  Gini coefficient. Almost perfect polarisation of web adoption in the
  early stages at a granular level More equally diffused at the Local
  Authority level Plateau overtime
\end{itemize}

S-shaped diffusion curves

\begin{itemize}
\item
  S for LADs per firm and OA. \textbf{TODO}: OA per firm? fix firms over
  time.
\item
  Fast and slow LAs map. There is clustering
\item
  \textbf{TODo} check s\_uk\_firm\_11
\item
  ranks: there is stability and movement
\end{itemize}

\hypertarget{sec3.2}{%
\subsection{3.2 Case study approach}\label{sec3.2}}

\hypertarget{sec4}{%
\section{Digital technologies and spatial structure: a global
view}\label{sec4}}

\hypertarget{sec5}{%
\section{The impact of ICT on the US and the UK spatial
structures}\label{sec5}}

\hypertarget{sec6}{%
\section{Discussion and conclusions}\label{sec6}}

contrary to results from future studies regarding social media
\citep{lengyel2020role}, web technologies did not exclusively spread
from a central location.


\renewcommand\refname{References}
  \bibliography{./paper/submission1/bibliography}


\end{document}
