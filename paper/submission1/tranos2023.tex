% Options for packages loaded elsewhere
\PassOptionsToPackage{unicode}{hyperref}
\PassOptionsToPackage{hyphens}{url}
\PassOptionsToPackage{dvipsnames,svgnames,x11names}{xcolor}
%
\documentclass[
  authoryear,
  preprint,
  3p]{elsarticle}

\usepackage{amsmath,amssymb}
\usepackage{iftex}
\ifPDFTeX
  \usepackage[T1]{fontenc}
  \usepackage[utf8]{inputenc}
  \usepackage{textcomp} % provide euro and other symbols
\else % if luatex or xetex
  \usepackage{unicode-math}
  \defaultfontfeatures{Scale=MatchLowercase}
  \defaultfontfeatures[\rmfamily]{Ligatures=TeX,Scale=1}
\fi
\usepackage{lmodern}
\ifPDFTeX\else  
    % xetex/luatex font selection
\fi
% Use upquote if available, for straight quotes in verbatim environments
\IfFileExists{upquote.sty}{\usepackage{upquote}}{}
\IfFileExists{microtype.sty}{% use microtype if available
  \usepackage[]{microtype}
  \UseMicrotypeSet[protrusion]{basicmath} % disable protrusion for tt fonts
}{}
\makeatletter
\@ifundefined{KOMAClassName}{% if non-KOMA class
  \IfFileExists{parskip.sty}{%
    \usepackage{parskip}
  }{% else
    \setlength{\parindent}{0pt}
    \setlength{\parskip}{6pt plus 2pt minus 1pt}}
}{% if KOMA class
  \KOMAoptions{parskip=half}}
\makeatother
\usepackage{xcolor}
\setlength{\emergencystretch}{3em} % prevent overfull lines
\setcounter{secnumdepth}{5}
% Make \paragraph and \subparagraph free-standing
\ifx\paragraph\undefined\else
  \let\oldparagraph\paragraph
  \renewcommand{\paragraph}[1]{\oldparagraph{#1}\mbox{}}
\fi
\ifx\subparagraph\undefined\else
  \let\oldsubparagraph\subparagraph
  \renewcommand{\subparagraph}[1]{\oldsubparagraph{#1}\mbox{}}
\fi


\providecommand{\tightlist}{%
  \setlength{\itemsep}{0pt}\setlength{\parskip}{0pt}}\usepackage{longtable,booktabs,array}
\usepackage{calc} % for calculating minipage widths
% Correct order of tables after \paragraph or \subparagraph
\usepackage{etoolbox}
\makeatletter
\patchcmd\longtable{\par}{\if@noskipsec\mbox{}\fi\par}{}{}
\makeatother
% Allow footnotes in longtable head/foot
\IfFileExists{footnotehyper.sty}{\usepackage{footnotehyper}}{\usepackage{footnote}}
\makesavenoteenv{longtable}
\usepackage{graphicx}
\makeatletter
\def\maxwidth{\ifdim\Gin@nat@width>\linewidth\linewidth\else\Gin@nat@width\fi}
\def\maxheight{\ifdim\Gin@nat@height>\textheight\textheight\else\Gin@nat@height\fi}
\makeatother
% Scale images if necessary, so that they will not overflow the page
% margins by default, and it is still possible to overwrite the defaults
% using explicit options in \includegraphics[width, height, ...]{}
\setkeys{Gin}{width=\maxwidth,height=\maxheight,keepaspectratio}
% Set default figure placement to htbp
\makeatletter
\def\fps@figure{htbp}
\makeatother

\makeatletter
\makeatother
\makeatletter
\makeatother
\makeatletter
\@ifpackageloaded{caption}{}{\usepackage{caption}}
\AtBeginDocument{%
\ifdefined\contentsname
  \renewcommand*\contentsname{Table of contents}
\else
  \newcommand\contentsname{Table of contents}
\fi
\ifdefined\listfigurename
  \renewcommand*\listfigurename{List of Figures}
\else
  \newcommand\listfigurename{List of Figures}
\fi
\ifdefined\listtablename
  \renewcommand*\listtablename{List of Tables}
\else
  \newcommand\listtablename{List of Tables}
\fi
\ifdefined\figurename
  \renewcommand*\figurename{Figure}
\else
  \newcommand\figurename{Figure}
\fi
\ifdefined\tablename
  \renewcommand*\tablename{Table}
\else
  \newcommand\tablename{Table}
\fi
}
\@ifpackageloaded{float}{}{\usepackage{float}}
\floatstyle{ruled}
\@ifundefined{c@chapter}{\newfloat{codelisting}{h}{lop}}{\newfloat{codelisting}{h}{lop}[chapter]}
\floatname{codelisting}{Listing}
\newcommand*\listoflistings{\listof{codelisting}{List of Listings}}
\makeatother
\makeatletter
\@ifpackageloaded{caption}{}{\usepackage{caption}}
\@ifpackageloaded{subcaption}{}{\usepackage{subcaption}}
\makeatother
\makeatletter
\@ifpackageloaded{tcolorbox}{}{\usepackage[skins,breakable]{tcolorbox}}
\makeatother
\makeatletter
\@ifundefined{shadecolor}{\definecolor{shadecolor}{rgb}{.97, .97, .97}}
\makeatother
\makeatletter
\makeatother
\makeatletter
\makeatother
\journal{Journal Name}
\ifLuaTeX
  \usepackage{selnolig}  % disable illegal ligatures
\fi
\usepackage[]{natbib}
\bibliographystyle{elsarticle-harv}
\IfFileExists{bookmark.sty}{\usepackage{bookmark}}{\usepackage{hyperref}}
\IfFileExists{xurl.sty}{\usepackage{xurl}}{} % add URL line breaks if available
\urlstyle{same} % disable monospaced font for URLs
\hypersetup{
  pdftitle={Digital economy in the UK: a multi-scalar story of the diffusion of web technologies},
  pdfauthor={Emmanouil Tranos},
  pdfkeywords={keyword1, keyword2},
  colorlinks=true,
  linkcolor={blue},
  filecolor={Maroon},
  citecolor={Blue},
  urlcolor={Blue},
  pdfcreator={LaTeX via pandoc}}

\setlength{\parindent}{6pt}
\begin{document}

\begin{frontmatter}
\title{Digital economy in the UK: a multi-scalar story of the diffusion
of web technologies}
\author[1]{Emmanouil Tranos%
\corref{cor1}%
}
 \ead{e.tranos@bristol.ac.uk} 

\affiliation[1]{organization={University of Bristol and The Alan Turing
Institute},postcode={UK},postcodesep={}}

\cortext[cor1]{Corresponding author}

        
\begin{abstract}
This paper maps the participation in the digital economy and its
evolution in the UK over space and time. Most of the existing economic
geography literature which dealt with the spatiality of the internet
employed supply-side measures, such as infrastructural capacity, in
order to understand the geography of the digital economy and its
potential spatial economic effects. Useful as these approaches might
have been, they cannot capture the micro-processes and the
characteristics of the individual online behaviour. Using large volumes
of archived and geolocated web content, this paper models the diffusion
of web technologies over space and time in the UK. The data and
geolocation strategy allow to capture these processes at a very granular
spatial scale. The modelling approach, which is based on simple spatial
analytical methods and on the estimation of diffusion curves at various
scales, enables to depict the role of geography and other cognitive
factors which drove the diffusion of web technologies. Although the
focus is on a recent historical period -- 1996-2012 -- the results of
the analysis depict diffusion mechanisms which can be very useful in
understanding the evolutionary patterns of the adoption of other newer
technologies.
\end{abstract}





\begin{keyword}
    keyword1 \sep 
    keyword2
\end{keyword}
\end{frontmatter}
    \ifdefined\Shaded\renewenvironment{Shaded}{\begin{tcolorbox}[breakable, frame hidden, boxrule=0pt, interior hidden, borderline west={3pt}{0pt}{shadecolor}, enhanced, sharp corners]}{\end{tcolorbox}}\fi

\hypertarget{sec1}{%
\section{Introduction}\label{sec1}}

Geographers were always interested in how new technologies and
innovations diffuse across space and time and, importantly, how such
spatio-temporal processes can be modelled. After all, diffusion together
with invention and innovation are considered the pillars of
technological change \citep{das2022diffusion}. The seminal contribution
of \citet{hagerstrand1968innovation} is illustrative of this early
interest. However, the torch of exploring and modelling such processes
had been passed to other disciplines such as economics, business studies
and sociology well before the `cultural turn' of economic geography
\citep{perkins2005international}. A potential explanation of the lack of
geographical studies exploring the diffusion of new and, more
specifically for this paper, digital technologies across \emph{both}
space and time can be attributed to the scarcity of relevant and
granular enough data. As \citet{zook2022mapping} highlight, digital
activities are hardly ever captured in official data.

This paper offers such a contribution: a geographical study illustrating
how a new technology that is the web diffused over space and time in the
UK at a high level of spatial granularity during the \(1996-2012\)
period. It does so by employing a novel source of big data which
captures the active engagement with web technologies during that period.
By addressing this empirical question this paper exemplifies how the
combination of data sources which escape the traditional social science
domain and adequate research methods can offer new lenses to
geographical research regarding the understanding of technological
diffusion.

The motivation for this paper lies in the fact that there are various
stakeholders who are interested in knowing how new digital technologies
diffused over space and time and use this knowledge to make predictions
regarding the diffusion of related \emph{future} technologies. As per
\citet{leibowicz2016representing}, historical studies agree on the fact
that ``technologies diffuse at different times, at different rates, and
to different extents in different places, and can be significantly
influenced by policies'' \citep{victor1993}. \citet{meade2021modelling}
highlight that a variety of actors have a direct interest in gaining
such knowledge including network equipment suppliers; network operators,
regulatory and local authorities. These processes and their effects vary
a lot across scales: although the diffusion of a new technology might
not be optimal at a local level, it might be beneficial from a global
perspective as it could lead to faster diffusion to less advantaged
places \citep{leibowicz2016representing}. Despite the spatial
heterogeneity of such diffusion mechanisms and the policy relevance,
there are very limited attempts in the literature to analyse the
diffusion of new digital technologies at a detailed geographical level.

Technological diffusion, which is by definition an aggregated process,
can be discussed in parallel with individual adoption mechanisms. On the
one hand, \citet{rogers2010diffusion} identifies early adopter of new
technologies as `knowledgeable risk takers' and \citet{griliches1957} as
`profit maximisers' \citep{ding2010modeling}. Such individual agents are
rewarded because of their attitude towards new technologies and
innovations. On the other hand, \citet{perkins2011internet} attribute
diffusion to two processes: (i) epidemic-like mechanisms, which are
governed by distance, proximity and social interactions, and (ii) by
economic mechanisms as new innovations are adopted by users as they
become more profitable, valueable and useful.

This paper focuses on the diffusion of the web as new technology during
the \(1996-2012\) period. This was an exciting period for digital
technologies as it corresponds with the commercialisation of the
internet and, consequently, its almost universal adoption. The reader is
reminded that it was only in 1994 when Netscape Navigator was
introduced, a year before Microsoft's Internet Explorer \footnote{\url{https://www.theguardian.com/global/2015/mar/22/web-browser-came-back-haunt-microsoft}}.
Also, only \(9\) per cent of UK's households had access to the internet
in \(1998\) \citep{ons2018}, the web included mostly static webpages,
there were no social media and web browsing was happening exclusively
from desktop PCs as there were no smartphones \citep{tranosuk}. Hence,
it is fair to say that the study period captures the very early stages
of the diffusion on a new technology that is the web. This is key point
in the lifecycle of a new technology. Firstly, during this period new
technologies are expensive, crude and imperfect
\citep{rosenberg1994exploring, wilson201281}. A simple comparison
between Web 1.0 and Web 3.0 applications clearly illustrates this
\citep{tranos2020social}. During this period the performance of a new
technology is the main attraction and not the cost to access and use it
\citep{wilson2011lessons} There is a broader theoretical discussion in
the literature regarding the motivation behind early adoption. As
summarised by \citet{perkins2005international}, on the one hand,
epidemic models highlight the role of interpersonal contacts as a way
for new technologies to diffuse. On the other hand, economic models
underline the importance of heterogeneity. Different firms have
different structures and business plans, which define the potential
economic returns of the adoption of a new technology and, therefore, the
choice to adopt a new technology becomes an individual option. From a
broader and evolutionary perspective, initial conditions are essential
for the creation and evolution of path-dependent technological
development trajectories \citep{neffke2011regions, simmie2014new}. This
argument is even more relevant when the focus is on Information and
Communication Technologies (ICT) because of the commonly found lag
between investment and economic returns as reflected in the Solow
paradox \citep{acemoglu2014return, brynjolfsson2018artificial}.

Importantly, the data used here depicts the \emph{active} engagement
with the web in the UK as it contains all the commercial websites
archived by the Internet Archive \footnote{\href{See\%20https://archive.org/}{https://archive.org/}.}
that include a mention to at least one valid UK postcode in the web
text. Websites are identified as commercial ones if they are part of the
.co.uk second level domain (SLD). Also, the act of of creating a website
is understood here as active engagement with the web vis-à-vis the more
passive act of browsing the web or having an internet connection
\citep{tranosuk}. Previous studies have focused mostly on more passive
notions of engaging with digital technologies such as internet adoption
and internet speeds
\citep[e.g.][]{blank2018local, riddlesden2014broadband}. More details
about the data and the data generation process can be found in Section
\protect\hyperlink{sec3}{XX}. \textbf{ADD DATA DESCRIPTION, WHY .CO.UK}

\citet{grubler1990rise} Later Hagerstrand conceptualized physical
``barrier'' effects like lakes or uninhabited areas, which, in addition
to distance, act as further retarding effects on diffusion. These are
formalized in the form of ``zero'' or ``half'' contact multiplicators on
the (distance decaying) message flows.

\citet{grubler1990rise} With respect to the formalization of the
communication flows Hagerstrand defines a ``mean information field''
(MIF), in which the probability of communication is a negative function
of distance between individuals

\citet{wilson201281} Logistic growth describes an initial period of
gradual diffusion as a technology is introduced as a new commercial
application, moving then through a rapid, exponential growth phase,
before slowing and eventually saturating \citep{grubler1999dynamics}.
The substitution of incumbent technologies by new competitors leads to
subsequent decline and eventual obsolescence.

\hypertarget{sec2}{%
\section{Literature review}\label{sec2}}

Geographical diffusion is a synthesis of different processes. On the one
hand, we can identify purely spatial or, in other words, contagious
processes. Adjacency and, more broadly, distance are the key drivers of
diffusion. This perspective draws similarities with epidemics:
innovation just like pathogens spreads because of contagion and,
consequently, proximity and exposure \citep{hivner2003facilitating}. On
the other hand, we can identify hierarchical processes. Instead of
horizontal distance-based diffusion mechanisms, the top-down hierarchy
of urban systems drives diffusion. In reality, the synthesis of these
two processes represents how new technologies diffuse over space and
time \citep{morrill2020spatial}.

These ideas were firstly introduced by Torsten Hägerstrand and his
thesis entitled `Innovation Diffusion as a Spatial Process'
\citep{hagerstrand1968innovation}. Hägerstrand was the first one to
identify diffusion as a geographical process. The starting point was the
idea that diffusion is based on passing information through social
networks, which themselves tend to be defined by geography. Hence, he
identified the `neighbourhood' effect of how information, and
consequently, innovation diffuse. He used agricultural innovations to
test and model his ideas using Monte Carlo simulations. Hägerstrand also
incorporated the role of hierarchy and how some phenomena maybe firstly
adopted in larger cities and then diffuse to second tier ones. This is a
sequential instead of a simultaneous process, which resembles the
`lead-lag' spatial acceleration effect in market research
\citep{bento2018time, PERES201091}. Hägerstrand is more widely known
though for highlighting the role time plays in the diffusion of
innovations: an early-pioneering period, a middle fast accelerating
period and a final saturation one \citep{morrill2020spatial}.

The temporal dimension was further explored by Everett Rogers and his
seminal work on `Diffusion of Innovations' \citep{rogers2010diffusion}.
Rogers being a sociologist, his work focused not on the diffusion of
innovations over space and time, but instead on the adoption of new
technologies and innovations by individuals and the individual
mechanisms that drive the decisions behind adoption. He identified five
groups of individuals regarding their adoption speed: innovators, early
adopters, early majority, late majority and laggards. Key mechanism of
diffusion and adoption is always communication and how knowledge is
transferred within a social system. Therefore, all approaches agree on
the S-shaped diffusion and cumulative adoption pattern
\citep{grubler1990rise}.

Schmidt's Law empirically illustrates a similar pattern. \emph{Core} and
usually highly agglomerated regions is where new technologies are
invented and commercially deployed {[}\citet{grubler1990rise};
grubler1990rise{]}. This is where the first adopters tend to be based.
Then, technologies spread to the \emph{rim} and eventually to the
\emph{periphery}. Although adoption pace might be higher when new
technologies finally arrive to the periphery, the saturation levels
there may never reach the ones in the core because of the lack of
infrastructure or other necessary institutions
\citep{leibowicz2016representing}. \citet{grubler1990rise} effectively
summarises the three key characteristics of the spatial diffusion
process: (i) the cumulative level of adoption follows an S-shaped
pattern just like purely temporal models; (ii) diffusion is shaped by a
hierarchy effect in a form of a centrifugal force: from core to
periphery; and (iii) diffusion is also shaped by distance and a
neighbourhood effect and contaminate nearby locations.

The remaining of this section reviews empirical studies which analysed
the diffusion on new technologies over space and time. Although the
spatial dimension is present in most of the following studies, the level
of spatial detail is almost always more coarse than the one adopted
here. \citet{beardsell1999spatial} studied the evolution of the computer
industry in \(317\) US metro areas during the \(1977-1992\) period using
employment data. Their analysis indicated that the relative size
distribution holds for urban computer employment and also urban
heterogeneity is essential in explaining this distribution. in a recent
study, \citet{bednarz2020pulled} focused on wind turbines and modelled
their spatial diffusion across \(402\) German regions during
\(1970-2015\). Their key finding is that local demand than local supply
was the main driving factor.

At a global scale \citet{perkins2005international} explored whther the
diffusion rate of new technologies is driven by a latecomer advantage
and the engagement with the global economy via foreign direct
investments and trade. Their results illustrate that indeed latecomers
and developing countries experience diffusion of new technologies more
rapidly than early adopter and developed countries. At the same scale,
\citet{perkins2011internet} explored whether the adoption of previous
communication technologies that is mail, telegrams and telephones was
shaped by similar socioeconomic factors as the internet. Their results
indicated common patterns regarding the drivers behind the adoption of
different communication technologies.

Turning to studies that share more technological and scalar similarities
with this paper, \citet{ding2010modeling} modelled the spatial diffusion
of mobile telecommunications across regions in China. Their analysis
indicated that socioeconomic characteristics are important determinants
of the timing, speed and the level of mobile diffusion within China.
Using data from a Hungarian online social network,
\citet{lengyel2020role} analysed its adoption and the churn at a very
granular spatial level. Their results are in agreement with early
theoretical and empirical contributions reviewed here: assortativity,
urban scaling and distance are the key drivers of spatial diffusion. At
a global scale \citet{PAPAGIANNIDIS2015308} modelled the diffusion of
different web technologies technologies and practices. Interestingly,
they did so by using similar, but less extensive data as the one used
here. Their analysis illustrated how the diffusion of different web
technologies and practices follow an S-shaped pattern as well as the
different diffusion rates of the different technologies and practices.

\hypertarget{sec3}{%
\section{Materials and Methods}\label{sec3}}

For instance, we can approximate digital economic activities in a region
by measuring the volume of commercial webpages by considering only
.co.uk websites, which are dedicated to commercial activities (Thelwall,
2000). Such commercial websites are used to exchange information,
support online transactions and share opinions (Blazquez \& Domenech,
2018). Although nothing prevents a UK-based company from adopting a
generic TLD such as.com and, indeed, such cases escape our data, we do
not expect that such omissions could affect our results given the
popularity of the .uk ccTLD: UK consumers prefer to visit a .uk website
when they are searching for products or services (Hope, 2017); and
anecdotal evidence indicates that during the first half of 2000, three
.co.uk domains were registered every minute (OECD, 2001). Also, as Table
1 demonstrates, .co.uk was the most popular SLD under the .uk ccTLD in
2000. Hence, we use the total volume of all the archived webpages in a
region as a proxy for the level of digitization. We start the analysis
by focusing on commercial webpages (.co.uk).

\citet{wilson201281} growth function description in p.~86

from R: Asym/(1+exp((xmid-input)/scal)) using the terms from Wilson k/(1
+ e\^{}(t\_0 - t)/scal)

Wilson: k(1 + e\^{}-b(t - t\_0)) So, scal = - 1/b

The curve is symmetric* around t\_0

\textbf{minimum R2 = 95\%}, see also \citet{grubler1990rise}

The literature usually uses the saturation level as the asymptote. I am
using the total number of websites as we cannot compute a rate.

Moran's I as \citet{ding2010modeling} \textbf{TODO} for t\_0, diffusion
speed

\hypertarget{results}{%
\section{Results}\label{results}}

\hypertarget{old-ideas}{%
\subsection{Old ideas}\label{old-ideas}}

Hexagon density maps: reflect the spatial structure of Britain. Websites
are associated in places where people live and work.

\textbf{TODO} maps at the local authority level per firms and discuss
patterns

Neighbourhood effect: diffusion proceeds outwards from innovation
centers, first ``hitting'' nearby rather than far-away locations
(Grubler 1990)

\begin{itemize}
\item
  Moran's I: for OA and LAD over time
\item
  LISA maps: for OA and LAD over time More and less expected clusters.
  Different scales show different results
\item
  Website density regressions: for OA and LAD over time \textbf{TODO}
  add 0s Similar pattern
\end{itemize}

Hierarchy effect: from main centers to secondary ones -- central places

\begin{itemize}
\tightlist
\item
  Gini coefficient. Almost perfect polarisation of web adoption in the
  early stages at a granular level More equally diffused at the Local
  Authority level Plateau overtime
\end{itemize}

S-shaped diffusion curves

\begin{itemize}
\item
  S for LADs per firm and OA.
\item
  Fast and slow LAs map. There is clustering
\item
  s\_uk\_firm\_11
\item
  ranks: there is stability and movement
\end{itemize}

\hypertarget{new-structure}{%
\subsection{New structure}\label{new-structure}}

\begin{enumerate}
\def\labelenumi{\arabic{enumi}.}
\item
  S-shaped diffusion curves: S for LADs per firm. UK, fast/slow/examples
\item
  ranks: there is stability and movement
\item
  Neighbourhood effect: diffusion proceeds outwards from innovation
  centers, first ``hitting'' nearby rather than far-away locations
  (Grubler
\end{enumerate}

\begin{enumerate}
\def\labelenumi{\arabic{enumi})}
\setcounter{enumi}{1989}
\tightlist
\item
\end{enumerate}

\begin{itemize}
\item
  Moran's I: for OA and LAD over time
\item
  LISA maps: for OA and LAD over time More and less expected clusters.
  Different scales show different results
\item
  Website density regressions: for OA and LAD over time
\end{itemize}

\begin{enumerate}
\def\labelenumi{\arabic{enumi}.}
\setcounter{enumi}{3}
\tightlist
\item
  Hierarchy effect: from main centers to secondary ones -- central
  places
\end{enumerate}

\begin{itemize}
\tightlist
\item
  Gini coefficient. Almost perfect polarisation of web adoption in the
  early stages at a granular level More equally diffused at the Local
  Authority level Plateau overtime
\end{itemize}

\begin{enumerate}
\def\labelenumi{\arabic{enumi}.}
\setcounter{enumi}{4}
\tightlist
\item
  RF
\end{enumerate}

\begin{itemize}
\tightlist
\item
  ideal: (i) train RF for all years and all (1) LADs and (2) OAs with
  CAST and report metrics. (ii) train for all years and all but one
  region for (1) LADs and (2) OAs to predict to the holdout region.
  Reports predictions as region similarities.
\end{itemize}

\hypertarget{rf-results}{%
\subsubsection{RF results}\label{rf-results}}

The next section incorporates the above discussed spatial drivers of the
diffusion of web technologies into the same modelling framework. The aim
is to use variables depicting these spatial processes in order to
predict the diffusion of web technologies in the UK over space and time
and across different scales. Specifically, four different models are
estimated. Firstly, all the data points at the OA and LAD are utilised
in order to build RF models and assess their capacity to predict the
adoption of web technologies. These two models will reveal the
predictive capacity of the spatial processes and also allow to see how
the importance of such variables changes across scales. The next two
models will be again trained on web difusion at the two working scales:
OA and LAD. However, instead of using all the data points, the OA and
the LAD from one of the 12 UK regions are hold out and then the trained
model is used to predict web adoption in the OA or the LAD of the
held-out region. This process takes place recursively for all UK
regions. The difference in the predictive capacity of the different
samples will reveal how different are these spatial process across
regions and, importantly, at different scales.

It needs to be highlighted here that the cross-validation for all models
presented here is spatially and temporally sensitive. Instead of using
10 random folds, we employ the \texttt{CAST} package which allows holds
back data points from specific years and spatial units and use them for
testing in order to estimate the model performance
\citep{meyer2018improving}.

The models need to include variables that capture the three processes
that the relevant literature and the descriptive analysis presented in
the previous section highlighted. Namely, the models capture: (i) a
hierarchy effect with diffusion running from main centres to secondary
ones, (ii) a neighborhood effect according to which diffusion first hits
nearby locations, and (iii) the rather canonical pattern of the
diffusion over time as reflected in the S-shaped pattern in the
cumulative level of adoption.

To capture the hierarchy effect the models include as predictors a one
year lag of website density in London -- the largest city in the UK, a
one year lag of the website density in the nearest city and the same for
the nearest retail centre. Due to the small sizes of the retail centres,
the latter is only relevant for the OA-level models. In addition, the
models include the distance to London, the nearest city and the nearest
retail centre. The underlying logic is that the level of website
adoption in a spatial unit depends on the level of the adoption in
places further up in the urban hierarchy the previous year. To depict
the neighbourhood effect, the web density of the neighbouring spatial
units in the previous year is calculated. Again, the underpinning
rationale is that the level of web adoption within a spatial unit
depends on the level of web adoption in the neighbouring spatial units
the year before. This is the `hitting nearby locations first' argument.
Therefore, the spatial and temporal lag of the website density in LAD
and OA is calculated. Lastly, the time effect which is reflected on the
S curve for the cumulative adoption is captured by trend variable.
Hence, all four model will follow the following generic form
\ref{model}:

\begin{align} \label{model}
Website\,Density_{t} \sim Distance\,London +
Website\,density\,London_{t-1} +\notag\\
Distance\,Nearest\,City +
Website\,density\,Nearest\,City_{t-1} +\notag\\
Distance\,Nearest\,Retail_{i} +
Website\,density\,Nearest\,Retail_{t-1} +\notag\\
W*\, Website\,density_{t-1} +\notag\\ 
year_{t}
\end{align}

To assess the predictive capability of the model, three broadly utilised
metrics are employed: the coefficient of determination (R squared), mean
absolute error (MAE) and root mean square error (RMSE):

\begin{align}
R^2 = 1 - \frac{\sum_{k} (y_{k} - \hat{y_{k}})^2} {\sum_{k} (y_{k} - \overline{y_{k}})^2} \label{eq:rsquared}
\end{align}

\begin{align}
MAE = \frac{1}{N} \sum_{k = 1}^{N} |\hat{y_{k}} - y_{k}| \label{eq:mae}
\end{align}

\begin{align}
RMSE =  \sqrt{\frac{\sum_{k = 1}^{N} (\hat{y_{k}} - y_{k})^2} {N}} \label{eq:rmse}
\end{align}

\(y_{k}\) is the \(k^{th}\) observation of the dataset, which consists
of \(N\) observations in total. \(\hat{y_{k}}\) is the \(k_{th}\)
predicted value for the dependent variable and \(\overline{y_{k}}\) is
the average value of \(y\). The last two metrics are expressed in the
same units as the dependent variable -- websites per firm for the LAD
modes and the number of websites for the OA models -- while the first
one is the coefficient of determination between the observed and the
predicted values of website adoption. Regarding \(MAE\), it is the
absolute difference between the observed and the predicted website
adoption. While \(MAE\) does not penalise for large errors, \(RSME\)
does so as it is proportional to the squared difference between the
observed and the predicted trade flows. Hence larger errors weigh more
for \(RMSE\) \citep{pontius2008components}.

Table \ref{table.metrics.all} presents the model performance for the
first set on models, for which all data points are employed for training
and testing via cross validation. The first one is trained and tested on
374 LAD and the second on 232,296 OA for a 16 year period (1997-2012).
The results are remarkably good considering that the are the outcome of
space and time sensitive CV, so the the model does not suffer from
overfit. At the LAD level we the model predicts 81\% of the variation of
website density. Both error metrics indicate that the model error is
\((\frac{1}{20} - \frac{1}{30})\) of a website per firm. At the OA this
score drops down to 21\%. Considering its granularity, this is still a
remarkable performance. To contextualise it, the model results in a MAE
of one website for areas small enough to host less than 140
households.\footnote{According to the Office for National Statistics,
  80\% of OA in England and Wales host 110-139 households,
  \href{https://www.ons.gov.uk/census/2001censusandearlier/dataandproducts/outputgeography/outputareas}{www.ons.gov.uk}.}
Because of the small size of the spatial units, the distribution is
highly skewed and a significant part of them is not linked to any
websites. In 1997 only 1\% of the UK OA were associated with a website.
This should not come as a surprise as this was the very beginning of the
commercial internet and any activities with a digital footprint were
concentrated in a handful of areas. This was clearly illustrated in
Section \textbf{ADD}. At the end of the study period almost half of the
UK OA were not associated with a website. Again, given the granularity
of the data this should not come as a surprise. Despite the large
heterogeneity in website adoption, the model reached an R-Squared of
0.2.

\begin{longtable}[]{@{}llll@{}}
\caption{Model metrics \{\#table.metrics.all\}}\tabularnewline
\toprule\noalign{}
& RMSE & RSquared & MAE \\
\midrule\noalign{}
\endfirsthead
\toprule\noalign{}
& RMSE & RSquared & MAE \\
\midrule\noalign{}
\endhead
\bottomrule\noalign{}
\endlastfoot
Local Authorities & 0.032 & 0.810 & 0.019 \\
Output Areas & 5.000 & 0.205 & 1.047 \\
\end{longtable}

Figures \ref{var.imp.LAD} and \ref{var.imp.OA} plot the importance of
the difrerent predictors. When the focus is on the LAD, the website
density in the nearest city, in London and on the neighbouring LADs the
year before are the most important predictors. They are followed by the
yearly trend, while the spatial configuration as reflected on distances
to London or the nearest city only play a minor role. This can be
attributed to the rather coarse spatial scale of analysis. Nevertheless,
all previously discussed forces are at play in the diffusion of web
technologies at the LAD: the first two predictors depict the
hierarchical effect, the spatial and temporal lag of website density the
neighbourhood effect and the yearly trend the time-sensitive cumulative
adoption pattern.

When the much more granular scale of OA is adopted, the picture is
reversed. The most important predictors are the three distance variables
to London, the nearest city and the nearest retail centre. They stil
depict the hierarchical effect, but proximity to the different
population centres is more important than their lagged web densities in
predicting website diffusion. The neighbouring effect is less important
at this scale. What is interesting is the almost negligible role of the
yearly trend and London's website density. While the former probably
illustrates the large heterogeneity in how web technologies have been
adopted at this very fine scale, the later highlights that the
importance of past web adoption rates in large population centres is
surpassed by proximity to them and soatial configuration at this scale.

\begin{figure}

{\centering \includegraphics[width=1\textwidth,height=\textheight]{../../outputs/rf/figures/varimp_LA.png}

}

\caption{\label{var.imp.LAD}Variable importance, LAD}

\end{figure}

\begin{figure}

{\centering \includegraphics[width=1\textwidth,height=\textheight]{../../outputs/rf/figures/varimp_OA.png}

}

\caption{\label{var.imp.OA}Variable importance, OA}

\end{figure}

Table \ref{table.regions} presents the results of the recursive hold out
models which aim to highlight potential regional heteorigeinty of the
web diffusion spatial mechanisms. To begin with, as highlighted before,
there is a difference of magnitude of one order between the LAD and the
OA prediction errors, which is aligned with previous results that
employed all data points. What is of interest here is the regional
comparison. Table \ref{table.regions} illustrates some striking
similarities, but also a few significant differences. The regions the
web diffusion of which is better predicted using models trained in the
rest of the country are the same despite the scale of analysis: South
East, Wales, Yorkshire and The Humber and the North East of England. In
other words, these are the regions whose spatial diffusion of web
technologies is closer to the country's average. Despite the consistency
across scales, this is a diverse set of regions: \textbf{ADD
CHARACTERISTICS}.

At the other end of the spectrum, Scotland's and the North West's web
diffusion mechanisms are consistently diverging from the country's
average. This should not come as a surpise as these regions are
characterised of high levels of rurality and remoteness. Similarly,
London diffusion mechanisms diverge from the country's average and this
is consistent across scales. London's uniqueness in UK's urban system
and economy is also reflected in the spatial diffusion mechanisms of web
technologies within its LAD and OA. It needs to be highlighted though
that the difference between the RSquared of LAD and OA is more that an
order of magnitude signaling how difficult is to predict diffusion at
such a small spatial scale. Northern Ireland is an interesting case.
While it ranks at the bottom of the scale when the models are trained
and tested on LAD data, when the modelling adopts the more granular OA
scale, the spatial mechanisms that shape the web diffusion within this
region appear to be closer to the country's average. At this scale,
proximity or lack of in relative terms to the rest of the country become
a less important and the internal to the region spatial structure
predictors start playing a more important role \textbf{CHECK NI OA}.

\begin{longtable}[]{@{}
  >{\raggedright\arraybackslash}p{(\columnwidth - 8\tabcolsep) * \real{0.3731}}
  >{\raggedleft\arraybackslash}p{(\columnwidth - 8\tabcolsep) * \real{0.1940}}
  >{\raggedleft\arraybackslash}p{(\columnwidth - 8\tabcolsep) * \real{0.1343}}
  >{\raggedleft\arraybackslash}p{(\columnwidth - 8\tabcolsep) * \real{0.1791}}
  >{\raggedleft\arraybackslash}p{(\columnwidth - 8\tabcolsep) * \real{0.1194}}@{}}
\caption{Regional differences\label{table.regions}}\tabularnewline
\toprule\noalign{}
\begin{minipage}[b]{\linewidth}\raggedright
Region
\end{minipage} & \begin{minipage}[b]{\linewidth}\raggedleft
RSquared LAD
\end{minipage} & \begin{minipage}[b]{\linewidth}\raggedleft
Rank LAD
\end{minipage} & \begin{minipage}[b]{\linewidth}\raggedleft
RSquared OA
\end{minipage} & \begin{minipage}[b]{\linewidth}\raggedleft
Rank OA
\end{minipage} \\
\midrule\noalign{}
\endfirsthead
\toprule\noalign{}
\begin{minipage}[b]{\linewidth}\raggedright
Region
\end{minipage} & \begin{minipage}[b]{\linewidth}\raggedleft
RSquared LAD
\end{minipage} & \begin{minipage}[b]{\linewidth}\raggedleft
Rank LAD
\end{minipage} & \begin{minipage}[b]{\linewidth}\raggedleft
RSquared OA
\end{minipage} & \begin{minipage}[b]{\linewidth}\raggedleft
Rank OA
\end{minipage} \\
\midrule\noalign{}
\endhead
\bottomrule\noalign{}
\endlastfoot
South East & 0.947 & 1 & 0.134 & 2 \\
Wales & 0.916 & 2 & 0.131 & 3 \\
Yorkshire and The Humber & 0.906 & 3 & 0.144 & 1 \\
North East & 0.895 & 4 & 0.128 & 4 \\
West Midlands & 0.883 & 5 & 0.070 & 9 \\
East Midlands & 0.882 & 6 & 0.088 & 8 \\
East of England & 0.876 & 7 & 0.106 & 6 \\
South West & 0.864 & 8 & 0.117 & 5 \\
London & 0.805 & 9 & 0.055 & 10 \\
Scotland & 0.770 & 10 & 0.035 & 11 \\
North West & 0.664 & 11 & 0.017 & 12 \\
Nortern Ireland & 0.576 & 12 & 0.101 & 7 \\
\end{longtable}

\hypertarget{discussion-and-conclusions}{%
\section{Discussion and conclusions}\label{discussion-and-conclusions}}

contrary to results from future studies regarding social media
\citep{lengyel2020role}, web technologies did not exclusively spread
from a central location.


\renewcommand\refname{References}
  \bibliography{bibliography}


\end{document}
